% Options for packages loaded elsewhere
\PassOptionsToPackage{unicode}{hyperref}
\PassOptionsToPackage{hyphens}{url}
%
\documentclass[
]{article}
\usepackage{amsmath,amssymb}
\usepackage{lmodern}
\usepackage{iftex}
\ifPDFTeX
  \usepackage[T1]{fontenc}
  \usepackage[utf8]{inputenc}
  \usepackage{textcomp} % provide euro and other symbols
\else % if luatex or xetex
  \usepackage{unicode-math}
  \defaultfontfeatures{Scale=MatchLowercase}
  \defaultfontfeatures[\rmfamily]{Ligatures=TeX,Scale=1}
\fi
% Use upquote if available, for straight quotes in verbatim environments
\IfFileExists{upquote.sty}{\usepackage{upquote}}{}
\IfFileExists{microtype.sty}{% use microtype if available
  \usepackage[]{microtype}
  \UseMicrotypeSet[protrusion]{basicmath} % disable protrusion for tt fonts
}{}
\makeatletter
\@ifundefined{KOMAClassName}{% if non-KOMA class
  \IfFileExists{parskip.sty}{%
    \usepackage{parskip}
  }{% else
    \setlength{\parindent}{0pt}
    \setlength{\parskip}{6pt plus 2pt minus 1pt}}
}{% if KOMA class
  \KOMAoptions{parskip=half}}
\makeatother
\usepackage{xcolor}
\IfFileExists{xurl.sty}{\usepackage{xurl}}{} % add URL line breaks if available
\IfFileExists{bookmark.sty}{\usepackage{bookmark}}{\usepackage{hyperref}}
\hypersetup{
  pdftitle={Universidad Autónoma del Estado de México},
  hidelinks,
  pdfcreator={LaTeX via pandoc}}
\urlstyle{same} % disable monospaced font for URLs
\usepackage[margin=1in]{geometry}
\usepackage{color}
\usepackage{fancyvrb}
\newcommand{\VerbBar}{|}
\newcommand{\VERB}{\Verb[commandchars=\\\{\}]}
\DefineVerbatimEnvironment{Highlighting}{Verbatim}{commandchars=\\\{\}}
% Add ',fontsize=\small' for more characters per line
\usepackage{framed}
\definecolor{shadecolor}{RGB}{248,248,248}
\newenvironment{Shaded}{\begin{snugshade}}{\end{snugshade}}
\newcommand{\AlertTok}[1]{\textcolor[rgb]{0.94,0.16,0.16}{#1}}
\newcommand{\AnnotationTok}[1]{\textcolor[rgb]{0.56,0.35,0.01}{\textbf{\textit{#1}}}}
\newcommand{\AttributeTok}[1]{\textcolor[rgb]{0.77,0.63,0.00}{#1}}
\newcommand{\BaseNTok}[1]{\textcolor[rgb]{0.00,0.00,0.81}{#1}}
\newcommand{\BuiltInTok}[1]{#1}
\newcommand{\CharTok}[1]{\textcolor[rgb]{0.31,0.60,0.02}{#1}}
\newcommand{\CommentTok}[1]{\textcolor[rgb]{0.56,0.35,0.01}{\textit{#1}}}
\newcommand{\CommentVarTok}[1]{\textcolor[rgb]{0.56,0.35,0.01}{\textbf{\textit{#1}}}}
\newcommand{\ConstantTok}[1]{\textcolor[rgb]{0.00,0.00,0.00}{#1}}
\newcommand{\ControlFlowTok}[1]{\textcolor[rgb]{0.13,0.29,0.53}{\textbf{#1}}}
\newcommand{\DataTypeTok}[1]{\textcolor[rgb]{0.13,0.29,0.53}{#1}}
\newcommand{\DecValTok}[1]{\textcolor[rgb]{0.00,0.00,0.81}{#1}}
\newcommand{\DocumentationTok}[1]{\textcolor[rgb]{0.56,0.35,0.01}{\textbf{\textit{#1}}}}
\newcommand{\ErrorTok}[1]{\textcolor[rgb]{0.64,0.00,0.00}{\textbf{#1}}}
\newcommand{\ExtensionTok}[1]{#1}
\newcommand{\FloatTok}[1]{\textcolor[rgb]{0.00,0.00,0.81}{#1}}
\newcommand{\FunctionTok}[1]{\textcolor[rgb]{0.00,0.00,0.00}{#1}}
\newcommand{\ImportTok}[1]{#1}
\newcommand{\InformationTok}[1]{\textcolor[rgb]{0.56,0.35,0.01}{\textbf{\textit{#1}}}}
\newcommand{\KeywordTok}[1]{\textcolor[rgb]{0.13,0.29,0.53}{\textbf{#1}}}
\newcommand{\NormalTok}[1]{#1}
\newcommand{\OperatorTok}[1]{\textcolor[rgb]{0.81,0.36,0.00}{\textbf{#1}}}
\newcommand{\OtherTok}[1]{\textcolor[rgb]{0.56,0.35,0.01}{#1}}
\newcommand{\PreprocessorTok}[1]{\textcolor[rgb]{0.56,0.35,0.01}{\textit{#1}}}
\newcommand{\RegionMarkerTok}[1]{#1}
\newcommand{\SpecialCharTok}[1]{\textcolor[rgb]{0.00,0.00,0.00}{#1}}
\newcommand{\SpecialStringTok}[1]{\textcolor[rgb]{0.31,0.60,0.02}{#1}}
\newcommand{\StringTok}[1]{\textcolor[rgb]{0.31,0.60,0.02}{#1}}
\newcommand{\VariableTok}[1]{\textcolor[rgb]{0.00,0.00,0.00}{#1}}
\newcommand{\VerbatimStringTok}[1]{\textcolor[rgb]{0.31,0.60,0.02}{#1}}
\newcommand{\WarningTok}[1]{\textcolor[rgb]{0.56,0.35,0.01}{\textbf{\textit{#1}}}}
\usepackage{graphicx}
\makeatletter
\def\maxwidth{\ifdim\Gin@nat@width>\linewidth\linewidth\else\Gin@nat@width\fi}
\def\maxheight{\ifdim\Gin@nat@height>\textheight\textheight\else\Gin@nat@height\fi}
\makeatother
% Scale images if necessary, so that they will not overflow the page
% margins by default, and it is still possible to overwrite the defaults
% using explicit options in \includegraphics[width, height, ...]{}
\setkeys{Gin}{width=\maxwidth,height=\maxheight,keepaspectratio}
% Set default figure placement to htbp
\makeatletter
\def\fps@figure{htbp}
\makeatother
\setlength{\emergencystretch}{3em} % prevent overfull lines
\providecommand{\tightlist}{%
  \setlength{\itemsep}{0pt}\setlength{\parskip}{0pt}}
\setcounter{secnumdepth}{-\maxdimen} % remove section numbering
\usepackage{ragged2e}
\justifying
\ifLuaTeX
  \usepackage{selnolig}  % disable illegal ligatures
\fi

\title{Universidad Autónoma del Estado de México}
\usepackage{etoolbox}
\makeatletter
\providecommand{\subtitle}[1]{% add subtitle to \maketitle
  \apptocmd{\@title}{\par {\large #1 \par}}{}{}
}
\makeatother
\subtitle{Licenciatura en Actuaría}
\author{}
\date{\vspace{-2.5em}}

\begin{document}
\maketitle

\begin {centering}
\Huge \bfseries Análisis de la Mortalidad en México mediante el Modelo Lee-Carter: Integración de Datos Históricos y Proyecciones \par
\end {centering}
\vspace*{2cm}

\vspace{1cm}
\vspace{2cm}

\begin{centering}
\huge \bfseries 
Autores:


Fuentes Hinojosa Santiago de Jesus

Lopez Valerio Santiago \par
\end{centering}
\vspace*{2cm}

\begin{center}
\Large \textbf{Repositorio en GitHub:} \\
\href{https://github.com/SanctiYago/MODELO-LEE-CARTER-COMPLETO}{\Large \texttt{https://github.com/SanctiYago/MODELO-LEE-CARTER-COMPLETO}}
\end{center}

\vfill
\large  \par
\vfill

Toluca, Estado de México a 30 de junio de 2025

\clearpage
\raggedright

\clearpage

\tableofcontents
\clearpage

\justify

\hypertarget{objetivo}{%
\section{1. Objetivo}\label{objetivo}}

El objetivo principal de este trabajo es analizar la evolución de la
mortalidad en la población mexicana a lo largo de más de un siglo,
integrando datos históricos y recientes para aplicar el modelo
Lee-Carter. Con ello, se busca identificar tendencias en las tasas de
mortalidad y proyectar su comportamiento futuro, de esta manera se
pretende facilitar la toma de decisiones informadas en ámbitos
actuariales y de salud pública en México.

\hypertarget{introducciuxf3n}{%
\section{2. Introducción}\label{introducciuxf3n}}

En los estudios actuariales y demográficos, el análisis de la mortalidad
juega un papel importante para la toma de decisiones en políticas
públicas, planes de pensiones y seguros de vida. El presente proyecto
tiene como objetivo analizar las tendencias de mortalidad en México,
haciendo uso del modelo de Lee-Carter, uno de los métodos más
reconocidos para modelar y pronosticar tasas de mortalidad a través del
tiempo.

La relevancia de este estudio radica en la necesidad de entender cómo
han evolucionado las tasas de mortalidad por edad y sexo a lo largo de
más de un siglo, así como de contar con herramientas sólidas para
proyectar estos patrones en el futuro. De esta forma, se pueden estimar
expectativas de vida y riesgos actuariales con mayor precisión,
contribuyendo a una mejor planificación y gestión de recursos en áreas
como seguridad social, salud pública y seguros.

A lo largo del documento se abordarán los siguientes aspectos:

\textbf{1- Revisión de datos y preparación de la base:} Se describe el
origen de los datos, su limpieza y la integración de distintas fuentes
para conformar la base de datos final.

\textbf{2- Análisis descriptivo:} Se exploran las tasas de mortalidad
por edad y año para hombres y mujeres, presentando gráficas que ilustran
la evolución y los patrones más relevantes.

\textbf{3- Aplicación del modelo Lee-Carter:} Se explica la metodología
del modelo y su ajuste a los datos para ambos sexos, mostrando los
resultados y discutiendo su interpretación.

\textbf{4- Conclusiones:} Se presentan los hallazgos principales del
análisis y se discuten sus implicaciones en ámbitos demográficos y
actuariales, así como posibles líneas de investigación futura.

\hypertarget{revisiuxf3n-de-literatura}{%
\section{3. Revisión de Literatura}\label{revisiuxf3n-de-literatura}}

\hypertarget{mortality-forecasting-in-latin-america-an-assessment-of-the-lee-carter-model}{%
\subsection{1. Mortality Forecasting in Latin America: An Assessment of
the Lee-Carter
Model}\label{mortality-forecasting-in-latin-america-an-assessment-of-the-lee-carter-model}}

\href{https://www.tandfonline.com/doi/epdf/10.1080/03461238.2021.1928542?needAccess=true}{\emph{Mortality
Forecasting in Latin America: An Assessment of the Lee-Carter Model}}

Janssen, F., \& Debón, A. (2021). An age-specific evaluation of the
Lee-Carter mortality forecasting model. Scandinavian Actuarial Journal,
2021(8), 654-679. \url{https://doi.org/10.1080/03461238.2021.192854}

Los autores evalúan la aplicación del modelo Lee-Carter para pronosticar
la mortalidad en distintos países de América Latina. Su investigación
destaca los siguientes aspectos relevantes para este proyecto:

\textbf{1- Ventajas del modelo Lee-Carter:} Se enfatiza cómo este modelo
captura de forma efectiva la tendencia general de la mortalidad a través
del tiempo, descomponiendo las tasas de mortalidad en factores que
permiten aislar patrones de cambio a largo plazo.

\textbf{2- Adaptabilidad a distintas poblaciones:} El estudio muestra
que el enfoque Lee-Carter puede ajustarse a diversas realidades
demográficas, resultando útil para países con cambios acelerados en su
estructura etaria y en sus patrones de mortalidad.

\textbf{3- Importancia de la calidad de datos:} Se subraya la necesidad
de contar con series de mortalidad históricas consistentes y completas
para mejorar la fiabilidad de las proyecciones, algo que este proyecto
también busca asegurar al integrar múltiples fuentes de datos.

\textbf{4- Proyecciones a mediano y largo plazo:} El trabajo demuestra
que el modelo Lee-Carter ofrece una forma relativamente sencilla y
robusta de proyectar la evolución futura de la mortalidad, lo que lo
convierte en una herramienta valiosa para la planificación de políticas
públicas y la evaluación actuarial.

En conjunto, esta literatura respalda la pertinencia de emplear el
modelo Lee-Carter en el presente proyecto, al ofrecer evidencia empírica
de su eficacia y versatilidad para analizar y pronosticar tendencias de
mortalidad en contextos demográficos complejos.

\hypertarget{latin-american-mortality-database}{%
\subsection{2. Latin American Mortality
Database}\label{latin-american-mortality-database}}

\href{https://www.ssc.wisc.edu/cdha/latinmortality2/?page_id=28}{\emph{Latin
American Mortality Database}}

University of Wisconsin-Madison. (n.d.). Latin mortality project (2nd
ed.). Center for Demography and Ecology.
\url{https://www.ssc.wisc.edu/cdha/latinmortality2/}

Esta base de datos se utilizó para obtener las tablas de vida empleadas
en la estimación del modelo Lee-Carter. En ella se encuentran las tablas
de vida correspondientes a México clasificadas por sexo.

\hypertarget{the-latin-american-mortality-database-lambda.-methodological-document-version-ii-august-2021}{%
\subsection{3. The Latin American Mortality Database (LAMBdA).
Methodological Document Version II: August
2021}\label{the-latin-american-mortality-database-lambda.-methodological-document-version-ii-august-2021}}

Al momento de acceder a las tablas de vida, podemos encontrar esta
documentación la cual explica los métodos que siguen para estimar los
datos que tienen el base de datos. El capítulo 3 es el que se centra en
la parte de la estimación de las tablas de vida desde 1850-2020.

\textbf{1- Fuentes de datos:} se decriben las principales fuentes de
datos utilizados para la estimación de las tablas de vida, en los que se
incluyen censos de población y estadísticas vitales de defunciones.

Estas fuentes fueron la clave para construir las estimaciones
consistentes de la mortalidad, sin embargo, la disponibilidad de estos
datos varíaban a lo largo del tiempo y por países.

También relatan la complejidad a la que se enfrentaron, ya que una de
las principales limitaciones en la construcción de tablas de vida fue la
falta de datos precisos, especialmente antes de 1950. Esto debido a los
censos de población y a las estadísticas de mortalidad para cada país.

\textbf{2- Ajustes en las tasas de mortalidad:} para estimar las tasas
de mortalidad que vienen en las tablas de vida, las cuales son
fundamentales para el modelo Lee-Carter, se utilizaron métodos como el
modelo de ogiva generalizado de Coale y Demey (1967) y variantes de los
métodos de Brass.

\textbf{3- Mortalidad infantil:} en el caso de la mortalidad infantil y
juvenil, mencionan que hicieron énfasis especial en cómo estimar la
mortalidad infantil y niños pequeños, por ejemplo tasas como 5Q0 y 1Q0
en quellos países en donde los datos que se disponen son limitados.

\hypertarget{mortality-forecasting-at-age-65-and-above-an-age-specific-evaluation-of-the-lee-carter-model}{%
\subsection{4. Mortality forecasting at age 65 and above: an
age-specific evaluation of the Lee-Carter
model}\label{mortality-forecasting-at-age-65-and-above-an-age-specific-evaluation-of-the-lee-carter-model}}

Lee, R. D., \& Carter, L. (1992). Modeling and forecasting U.S.
mortality. Journal of the American Statistical Association, 87(419),
659-671. \url{https://doi.org/10.1080/03461238.2021.1928542}

\href{https://www.tandfonline.com/doi/full/10.1080/03461238.2021.1928542\#d1e148}{\emph{Mortality
forecasting at age 65 and above: an age-specific evaluation of the
Lee-Carter model}}

Este artículo analiza de manera profunda la precisión, el sesgo y la
robustez del modelo Lee-Carter (LC) para la proyección de mortalidad a
partir de los 65 años, evaluando su desempeño en cuatro países mediante
un enfoque fuera de la muestra que se utiliza. El estudio muestra que el
modelo LC es adecuado cuando la Tasa de Mejora de Mortalidad por Edad
(ASRMI) es constante, pero señala que las ASRMI rara vez son constantes
durante la práctica, lo que limita la precisión del modelo en la mayoría
de los casos.

\hypertarget{mortallity-modelling-and-forecasting-a-review-of-methods}{%
\subsection{5. Mortallity Modelling and forecasting: a review of
methods}\label{mortallity-modelling-and-forecasting-a-review-of-methods}}

\href{https://www.actuaries.org/CTTEES_TFM/Documents/Booth_Tickle.pdf}{\emph{Mortality
Modelling and forecasting: a review of methods}}

Booth, H., \& Tickle, L. (2008). Mortality modelling and forecasting: A
review of methods. Actuarial Association of Australia. Retrieved from
\url{https://www.actuaries.org/CTTEES_TFM/Documents/Booth_Tickle.pdf}

Este artículo proporciona una revisión exhaustiva de los métodos de
modelado y pronóstico de mortalidad, destacando las técnicas más comunes
que han sido utilizadas por los actuarios y demógrafos para pronosticar
las tasas de mortalidad y la esperanza de vida a través del tiempo.

Dentro de los métodos de modelado de mortalidad, se exploran avrias
técnicas de modelado, una de esras técnica es el modelo de Lee-Carter.
Se discute la ventaja de este modelo por su simplicidad y su
flexibilidad para capturar las tendencias de mejora de mortalidad a
través del tiempo.

También se puede observar que se habla sobre las limitaciones de este
modelo, este modelo no es capaz de captar los cambios abruptos de la
mortalidad.

\hypertarget{anuxe1lisis-exploratorio-de-datos}{%
\section{4. Análisis exploratorio de
datos}\label{anuxe1lisis-exploratorio-de-datos}}

\justifying

A continuación, se presentarán las librerías de R necesarias para el
desarrollo de este proyecto. Estas herramientas facilitarán la
manipulación y el análisis de datos, así como la generación de
resultados y visualizaciones de manera eficiente.

\begin{Shaded}
\begin{Highlighting}[]
\FunctionTok{library}\NormalTok{(corrplot)}
\FunctionTok{library}\NormalTok{(pacman)}
\FunctionTok{library}\NormalTok{(demography)}
\FunctionTok{library}\NormalTok{(ggplot2)}
\FunctionTok{library}\NormalTok{(reshape2)}
\FunctionTok{library}\NormalTok{(tidyverse)}
\FunctionTok{library}\NormalTok{(dplyr)}
\FunctionTok{library}\NormalTok{(skimr)}
\end{Highlighting}
\end{Shaded}

Después de haber cargado las librerías, procedimos a cargar la base de
datos correspondiente a la población masculina, comprendida entre los
años 1985 y 2005. Esta base de datos fue obtenida de la \emph{Latin
Mortality Database (Latin Mortality Database, n.d.)} recuperada de
\url{https://www.ssc.wisc.edu/cdha/latinmortality2/?page_id=28}.

\begin{Shaded}
\begin{Highlighting}[]
\NormalTok{datos\_lc5 }\OtherTok{\textless{}{-}} \FunctionTok{read.table}\NormalTok{(}\StringTok{"Mex\_mltpiv\_5x1.txt"}\NormalTok{, }\AttributeTok{header =}\NormalTok{ T, }\AttributeTok{sep =} \StringTok{""}\NormalTok{) }
\end{Highlighting}
\end{Shaded}

Para comenzar el análisis de nuestra base de datos, lo primero que
debemos hacer es examinar su estructura, lo que nos permitirá saber las
variables que contiene, su tipo y la forma en que se organizan.

\begin{Shaded}
\begin{Highlighting}[]
\FunctionTok{str}\NormalTok{(datos\_lc5)}
\end{Highlighting}
\end{Shaded}

\begin{verbatim}
## 'data.frame':    342 obs. of  10 variables:
##  $ ctry     : chr  "Mexico" "Mexico" "Mexico" "Mexico" ...
##  $ ctry.code: int  2310 2310 2310 2310 2310 2310 2310 2310 2310 2310 ...
##  $ year     : int  1895 1895 1895 1895 1895 1895 1895 1895 1895 1895 ...
##  $ age      : int  0 1 5 10 15 20 25 30 35 40 ...
##  $ mx       : num  0.30741 0.23314 0.0136 0.00742 0.00989 ...
##  $ qx       : num  0.2646 0.208 0.0657 0.0364 0.0483 ...
##  $ lx       : int  100000 73535 58242 54417 52435 49903 46862 43773 40780 37773 ...
##  $ Lx       : int  82268 253645 280694 267329 256098 242217 226896 211681 196684 181084 ...
##  $ Tx       : int  2872238 2789970 2536324 2255631 1988302 1732204 1489987 1263091 1051410 854726 ...
##  $ ex       : num  28.7 37.9 43.5 41.5 37.9 ...
\end{verbatim}

Podemos notar que tenemos 10 variables, de las cuales solo ``ctry'' es
de tipo caracter, las demás son numéricas. Es indispensable explicar que
significa cada variable:

-\textbf{mx}: tasa de mortalidad a edad x

-\textbf{qx}: probabilidad de muerte a edad x

-\textbf{lx}: número de sobrevivientes a edad x

-\textbf{Lx}: cantidad de años vividos entre las edades x y x+1

-\textbf{Tx}: años totales vividos

-\textbf{ex}: esperanza de vida a edad x

Luego, es fundamental describir nuestras variables: para las numéricas,
usaremos summary(), la cual nos dará estadísticas básicas (mínimo,
máximo, media, etc.). Estos pasos iniciales nos permiten asegurarnos de
que los datos están en orden y listos para un mejor análisis.

\begin{Shaded}
\begin{Highlighting}[]
\FunctionTok{summary}\NormalTok{(datos\_lc5)}
\end{Highlighting}
\end{Shaded}

\begin{verbatim}
##      ctry             ctry.code         year           age       
##  Length:342         Min.   :2310   Min.   :1895   Min.   : 0.00  
##  Class :character   1st Qu.:2310   1st Qu.:1910   1st Qu.:15.00  
##  Mode  :character   Median :2310   Median :1932   Median :40.00  
##                     Mean   :2310   Mean   :1939   Mean   :40.32  
##                     3rd Qu.:2310   3rd Qu.:1965   3rd Qu.:65.00  
##                     Max.   :2310   Max.   :2005   Max.   :85.00  
##        mx                qx                lx               Lx        
##  Min.   :0.00032   Min.   :0.00162   Min.   :   447   Min.   :  1072  
##  1st Qu.:0.01065   1st Qu.:0.04892   1st Qu.: 27385   1st Qu.: 98900  
##  Median :0.02164   Median :0.09072   Median : 49999   Median :217112  
##  Mean   :0.07351   Mean   :0.21238   Mean   : 50492   Mean   :216807  
##  3rd Qu.:0.09787   3rd Qu.:0.25204   3rd Qu.: 74606   3rd Qu.:300476  
##  Max.   :0.41755   Max.   :1.00000   Max.   :100000   Max.   :487300  
##        Tx                ex       
##  Min.   :   1072   Min.   : 2.39  
##  1st Qu.: 335332   1st Qu.:10.77  
##  Median :1275638   Median :25.78  
##  Mean   :1749231   Mean   :26.29  
##  3rd Qu.:2750162   3rd Qu.:38.00  
##  Max.   :7154615   Max.   :72.10
\end{verbatim}

Para este análisis, hemos optado por conservar únicamente las variables
que consideramos de interés, con el fin de enfocar nuestro estudio en
aquellos aspectos fundamentales de la base de datos. Al seleccionar
cuidadosamente estas variables, buscamos asegurar que cada una aporte
información relevante y contribuya de manera directa a la interpretación
precisa de los resultados que obtendremos más adelante.

\begin{Shaded}
\begin{Highlighting}[]
\NormalTok{datos\_lc2005 }\OtherTok{\textless{}{-}} \FunctionTok{select}\NormalTok{(datos\_lc5, year, age, mx, qx, lx, Lx, Tx, ex)}
\end{Highlighting}
\end{Shaded}

Nos hemos quedado unicamente con \textbf{year, age, mx, qx, lx, Lx, Tx,
ex.}

Hasta este punto, hemos constatado que los datos disponibles abarcan el
período comprendido entre 1895 y 2005. Con el objetivo de ampliar
nuestro análisis temporal, procederemos a incorporar información
adicional, concretamente, datos correspondientes al intervalo de 2005 a
2021. Esta ampliación se fundamenta en la información proporcionada en
la fuente \emph{Tablas de vida (UISP, n.d.),} disponible en
\url{https://uisp.insp.mx/wp/index.php/tablas-de-vida/}.

\begin{Shaded}
\begin{Highlighting}[]
\CommentTok{\# 2006 {-}{-}{-}{-}{-}{-}{-}{-}{-}{-}{-}{-}{-}{-}{-}{-}{-}{-}{-}{-}{-}{-}{-}{-}{-}{-}{-}{-}{-}{-}{-}{-}{-}{-}{-}{-}{-}{-}{-}{-}{-}{-}{-}{-}{-}{-}{-}{-}{-}{-}{-}{-}{-}{-}{-}{-}{-}{-}{-}{-}{-}{-}{-}{-}{-}{-}{-}{-}}

\NormalTok{tablas2006general }\OtherTok{\textless{}{-}} \FunctionTok{read.csv}\NormalTok{(}\StringTok{"2006.csv"}\NormalTok{, }\AttributeTok{sep=}\StringTok{"}\SpecialCharTok{\textbackslash{}t}\StringTok{"}\NormalTok{, }\AttributeTok{header=}\ConstantTok{TRUE}\NormalTok{, }\AttributeTok{fileEncoding=}\StringTok{"UTF{-}16LE"}\NormalTok{)}
\NormalTok{df3 }\OtherTok{\textless{}{-}}\NormalTok{ tablas2006general[}\SpecialCharTok{{-}}\DecValTok{1}\NormalTok{, ]}


\CommentTok{\# 2007 {-}{-}{-}{-}{-}{-}{-}{-}{-}{-}{-}{-}{-}{-}{-}{-}{-}{-}{-}{-}{-}{-}{-}{-}{-}{-}{-}{-}{-}{-}{-}{-}{-}{-}{-}{-}{-}{-}{-}{-}{-}{-}{-}{-}{-}{-}{-}{-}{-}{-}{-}{-}{-}{-}{-}{-}{-}{-}{-}{-}{-}{-}{-}{-}{-}{-}{-}{-}}

\NormalTok{tablas2007general }\OtherTok{\textless{}{-}} \FunctionTok{read.csv}\NormalTok{(}\StringTok{"2007\_1.csv"}\NormalTok{, }\AttributeTok{sep=}\StringTok{"}\SpecialCharTok{\textbackslash{}t}\StringTok{"}\NormalTok{, }\AttributeTok{header=}\ConstantTok{TRUE}\NormalTok{, }\AttributeTok{fileEncoding=}\StringTok{"UTF{-}16LE"}\NormalTok{)}
\NormalTok{df4 }\OtherTok{\textless{}{-}}\NormalTok{ tablas2007general[}\SpecialCharTok{{-}}\DecValTok{1}\NormalTok{, ]}


\CommentTok{\# 2008 {-}{-}{-}{-}{-}{-}{-}{-}{-}{-}{-}{-}{-}{-}{-}{-}{-}{-}{-}{-}{-}{-}{-}{-}{-}{-}{-}{-}{-}{-}{-}{-}{-}{-}{-}{-}{-}{-}{-}{-}{-}{-}{-}{-}{-}{-}{-}{-}{-}{-}{-}{-}{-}{-}{-}{-}{-}{-}{-}{-}{-}{-}{-}{-}{-}{-}{-}{-}}
\NormalTok{tablas2008general }\OtherTok{\textless{}{-}} \FunctionTok{read.csv}\NormalTok{(}\StringTok{"2008.csv"}\NormalTok{, }\AttributeTok{sep=}\StringTok{"}\SpecialCharTok{\textbackslash{}t}\StringTok{"}\NormalTok{, }\AttributeTok{header=}\ConstantTok{TRUE}\NormalTok{, }\AttributeTok{fileEncoding=}\StringTok{"UTF{-}16LE"}\NormalTok{)}
\NormalTok{df5 }\OtherTok{\textless{}{-}}\NormalTok{ tablas2008general[}\SpecialCharTok{{-}}\DecValTok{1}\NormalTok{, ]}


\CommentTok{\# 2009 {-}{-}{-}{-}{-}{-}{-}{-}{-}{-}{-}{-}{-}{-}{-}{-}{-}{-}{-}{-}{-}{-}{-}{-}{-}{-}{-}{-}{-}{-}{-}{-}{-}{-}{-}{-}{-}{-}{-}{-}{-}{-}{-}{-}{-}{-}{-}{-}{-}{-}{-}{-}{-}{-}{-}{-}{-}{-}{-}{-}{-}{-}{-}{-}{-}{-}{-}{-}}

\NormalTok{tablas2009general }\OtherTok{\textless{}{-}} \FunctionTok{read.csv}\NormalTok{(}\StringTok{"2009.csv"}\NormalTok{, }\AttributeTok{sep=}\StringTok{"}\SpecialCharTok{\textbackslash{}t}\StringTok{"}\NormalTok{, }\AttributeTok{header=}\ConstantTok{TRUE}\NormalTok{, }\AttributeTok{fileEncoding=}\StringTok{"UTF{-}16LE"}\NormalTok{)}
\NormalTok{df6 }\OtherTok{\textless{}{-}}\NormalTok{ tablas2009general[}\SpecialCharTok{{-}}\DecValTok{1}\NormalTok{, ]}


\CommentTok{\# 2010 {-}{-}{-}{-}{-}{-}{-}{-}{-}{-}{-}{-}{-}{-}{-}{-}{-}{-}{-}{-}{-}{-}{-}{-}{-}{-}{-}{-}{-}{-}{-}{-}{-}{-}{-}{-}{-}{-}{-}{-}{-}{-}{-}{-}{-}{-}{-}{-}{-}{-}{-}{-}{-}{-}{-}{-}{-}{-}{-}{-}{-}{-}{-}{-}{-}{-}{-}{-}}
\NormalTok{tablas2010general }\OtherTok{\textless{}{-}} \FunctionTok{read.csv}\NormalTok{(}\StringTok{"2010.csv"}\NormalTok{, }\AttributeTok{sep=}\StringTok{"}\SpecialCharTok{\textbackslash{}t}\StringTok{"}\NormalTok{, }\AttributeTok{header=}\ConstantTok{TRUE}\NormalTok{, }\AttributeTok{fileEncoding=}\StringTok{"UTF{-}16LE"}\NormalTok{)}
\NormalTok{df7 }\OtherTok{\textless{}{-}}\NormalTok{ tablas2010general[}\SpecialCharTok{{-}}\DecValTok{1}\NormalTok{, ]}


\CommentTok{\# 2011 {-}{-}{-}{-}{-}{-}{-}{-}{-}{-}{-}{-}{-}{-}{-}{-}{-}{-}{-}{-}{-}{-}{-}{-}{-}{-}{-}{-}{-}{-}{-}{-}{-}{-}{-}{-}{-}{-}{-}{-}{-}{-}{-}{-}{-}{-}{-}{-}{-}{-}{-}{-}{-}{-}{-}{-}{-}{-}{-}{-}{-}{-}{-}{-}{-}{-}{-}{-}}

\NormalTok{tablas2011general }\OtherTok{\textless{}{-}} \FunctionTok{read.csv}\NormalTok{(}\StringTok{"2011.csv"}\NormalTok{, }\AttributeTok{sep=}\StringTok{"}\SpecialCharTok{\textbackslash{}t}\StringTok{"}\NormalTok{, }\AttributeTok{header=}\ConstantTok{TRUE}\NormalTok{, }\AttributeTok{fileEncoding=}\StringTok{"UTF{-}16LE"}\NormalTok{)}
\NormalTok{df8 }\OtherTok{\textless{}{-}}\NormalTok{ tablas2011general[}\SpecialCharTok{{-}}\DecValTok{1}\NormalTok{, ]}


\CommentTok{\# 2012 {-}{-}{-}{-}{-}{-}{-}{-}{-}{-}{-}{-}{-}{-}{-}{-}{-}{-}{-}{-}{-}{-}{-}{-}{-}{-}{-}{-}{-}{-}{-}{-}{-}{-}{-}{-}{-}{-}{-}{-}{-}{-}{-}{-}{-}{-}{-}{-}{-}{-}{-}{-}{-}{-}{-}{-}{-}{-}{-}{-}{-}{-}{-}{-}{-}{-}{-}{-}}

\NormalTok{tablas2012general }\OtherTok{\textless{}{-}} \FunctionTok{read.csv}\NormalTok{(}\StringTok{"2012.csv"}\NormalTok{, }\AttributeTok{sep=}\StringTok{"}\SpecialCharTok{\textbackslash{}t}\StringTok{"}\NormalTok{, }\AttributeTok{header=}\ConstantTok{TRUE}\NormalTok{, }\AttributeTok{fileEncoding=}\StringTok{"UTF{-}16LE"}\NormalTok{)}
\NormalTok{df9 }\OtherTok{\textless{}{-}}\NormalTok{ tablas2012general[}\SpecialCharTok{{-}}\DecValTok{1}\NormalTok{, ]}


\CommentTok{\# 2013 {-}{-}{-}{-}{-}{-}{-}{-}{-}{-}{-}{-}{-}{-}{-}{-}{-}{-}{-}{-}{-}{-}{-}{-}{-}{-}{-}{-}{-}{-}{-}{-}{-}{-}{-}{-}{-}{-}{-}{-}{-}{-}{-}{-}{-}{-}{-}{-}{-}{-}{-}{-}{-}{-}{-}{-}{-}{-}{-}{-}{-}{-}{-}{-}{-}{-}{-}{-}}

\NormalTok{tablas2013general }\OtherTok{\textless{}{-}} \FunctionTok{read.csv}\NormalTok{(}\StringTok{"2013.csv"}\NormalTok{, }\AttributeTok{sep=}\StringTok{"}\SpecialCharTok{\textbackslash{}t}\StringTok{"}\NormalTok{, }\AttributeTok{header=}\ConstantTok{TRUE}\NormalTok{, }\AttributeTok{fileEncoding=}\StringTok{"UTF{-}16LE"}\NormalTok{)}

\NormalTok{df10 }\OtherTok{\textless{}{-}}\NormalTok{ tablas2013general[}\SpecialCharTok{{-}}\DecValTok{1}\NormalTok{, ]}


\CommentTok{\# 2014 {-}{-}{-}{-}{-}{-}{-}{-}{-}{-}{-}{-}{-}{-}{-}{-}{-}{-}{-}{-}{-}{-}{-}{-}{-}{-}{-}{-}{-}{-}{-}{-}{-}{-}{-}{-}{-}{-}{-}{-}{-}{-}{-}{-}{-}{-}{-}{-}{-}{-}{-}{-}{-}{-}{-}{-}{-}{-}{-}{-}{-}{-}{-}{-}{-}{-}{-}{-}}

\NormalTok{tablas2014general }\OtherTok{\textless{}{-}} \FunctionTok{read.csv}\NormalTok{(}\StringTok{"2014.csv"}\NormalTok{, }\AttributeTok{sep=}\StringTok{"}\SpecialCharTok{\textbackslash{}t}\StringTok{"}\NormalTok{, }\AttributeTok{header=}\ConstantTok{TRUE}\NormalTok{, }\AttributeTok{fileEncoding=}\StringTok{"UTF{-}16LE"}\NormalTok{)}
\NormalTok{df11 }\OtherTok{\textless{}{-}}\NormalTok{ tablas2014general[}\SpecialCharTok{{-}}\DecValTok{1}\NormalTok{, ]}


\CommentTok{\# 2015 {-}{-}{-}{-}{-}{-}{-}{-}{-}{-}{-}{-}{-}{-}{-}{-}{-}{-}{-}{-}{-}{-}{-}{-}{-}{-}{-}{-}{-}{-}{-}{-}{-}{-}{-}{-}{-}{-}{-}{-}{-}{-}{-}{-}{-}{-}{-}{-}{-}{-}{-}{-}{-}{-}{-}{-}{-}{-}{-}{-}{-}{-}{-}{-}{-}{-}{-}{-}}

\NormalTok{tablas2015general }\OtherTok{\textless{}{-}} \FunctionTok{read.csv}\NormalTok{(}\StringTok{"2015.csv"}\NormalTok{, }\AttributeTok{sep=}\StringTok{"}\SpecialCharTok{\textbackslash{}t}\StringTok{"}\NormalTok{, }\AttributeTok{header=}\ConstantTok{TRUE}\NormalTok{, }\AttributeTok{fileEncoding=}\StringTok{"UTF{-}16LE"}\NormalTok{)}
\NormalTok{df12 }\OtherTok{\textless{}{-}}\NormalTok{ tablas2015general[}\SpecialCharTok{{-}}\DecValTok{1}\NormalTok{, ]}


\CommentTok{\# 2016 {-}{-}{-}{-}{-}{-}{-}{-}{-}{-}{-}{-}{-}{-}{-}{-}{-}{-}{-}{-}{-}{-}{-}{-}{-}{-}{-}{-}{-}{-}{-}{-}{-}{-}{-}{-}{-}{-}{-}{-}{-}{-}{-}{-}{-}{-}{-}{-}{-}{-}{-}{-}{-}{-}{-}{-}{-}{-}{-}{-}{-}{-}{-}{-}{-}{-}{-}{-}}

\NormalTok{tablas2016general }\OtherTok{\textless{}{-}} \FunctionTok{read.csv}\NormalTok{(}\StringTok{"2016.csv"}\NormalTok{, }\AttributeTok{sep=}\StringTok{"}\SpecialCharTok{\textbackslash{}t}\StringTok{"}\NormalTok{, }\AttributeTok{header=}\ConstantTok{TRUE}\NormalTok{, }\AttributeTok{fileEncoding=}\StringTok{"UTF{-}16LE"}\NormalTok{)}
\NormalTok{df13 }\OtherTok{\textless{}{-}}\NormalTok{ tablas2016general[}\SpecialCharTok{{-}}\DecValTok{1}\NormalTok{, ]}


\CommentTok{\# 2017 {-}{-}{-}{-}{-}{-}{-}{-}{-}{-}{-}{-}{-}{-}{-}{-}{-}{-}{-}{-}{-}{-}{-}{-}{-}{-}{-}{-}{-}{-}{-}{-}{-}{-}{-}{-}{-}{-}{-}{-}{-}{-}{-}{-}{-}{-}{-}{-}{-}{-}{-}{-}{-}{-}{-}{-}{-}{-}{-}{-}{-}{-}{-}{-}{-}{-}{-}{-}}

\NormalTok{tablas2017general }\OtherTok{\textless{}{-}} \FunctionTok{read.csv}\NormalTok{(}\StringTok{"2017.csv"}\NormalTok{, }\AttributeTok{sep=}\StringTok{"}\SpecialCharTok{\textbackslash{}t}\StringTok{"}\NormalTok{, }\AttributeTok{header=}\ConstantTok{TRUE}\NormalTok{, }\AttributeTok{fileEncoding=}\StringTok{"UTF{-}16LE"}\NormalTok{)}
\NormalTok{df14 }\OtherTok{\textless{}{-}}\NormalTok{ tablas2017general[}\SpecialCharTok{{-}}\DecValTok{1}\NormalTok{, ]}


\CommentTok{\# 2018 {-}{-}{-}{-}{-}{-}{-}{-}{-}{-}{-}{-}{-}{-}{-}{-}{-}{-}{-}{-}{-}{-}{-}{-}{-}{-}{-}{-}{-}{-}{-}{-}{-}{-}{-}{-}{-}{-}{-}{-}{-}{-}{-}{-}{-}{-}{-}{-}{-}{-}{-}{-}{-}{-}{-}{-}{-}{-}{-}{-}{-}{-}{-}{-}{-}{-}{-}{-}}

\NormalTok{tablas2018general }\OtherTok{\textless{}{-}} \FunctionTok{read.csv}\NormalTok{(}\StringTok{"2018.csv"}\NormalTok{, }\AttributeTok{sep=}\StringTok{"}\SpecialCharTok{\textbackslash{}t}\StringTok{"}\NormalTok{, }\AttributeTok{header=}\ConstantTok{TRUE}\NormalTok{, }\AttributeTok{fileEncoding=}\StringTok{"UTF{-}16LE"}\NormalTok{)}
\NormalTok{df15 }\OtherTok{\textless{}{-}}\NormalTok{ tablas2018general[}\SpecialCharTok{{-}}\DecValTok{1}\NormalTok{, ]}


\CommentTok{\# 2019 {-}{-}{-}{-}{-}{-}{-}{-}{-}{-}{-}{-}{-}{-}{-}{-}{-}{-}{-}{-}{-}{-}{-}{-}{-}{-}{-}{-}{-}{-}{-}{-}{-}{-}{-}{-}{-}{-}{-}{-}{-}{-}{-}{-}{-}{-}{-}{-}{-}{-}{-}{-}{-}{-}{-}{-}{-}{-}{-}{-}{-}{-}{-}{-}{-}{-}{-}{-}}

\NormalTok{tablas2019general }\OtherTok{\textless{}{-}} \FunctionTok{read.csv}\NormalTok{(}\StringTok{"2019.csv"}\NormalTok{, }\AttributeTok{sep=}\StringTok{"}\SpecialCharTok{\textbackslash{}t}\StringTok{"}\NormalTok{, }\AttributeTok{header=}\ConstantTok{TRUE}\NormalTok{, }\AttributeTok{fileEncoding=}\StringTok{"UTF{-}16LE"}\NormalTok{)}
\NormalTok{df16 }\OtherTok{\textless{}{-}}\NormalTok{ tablas2019general[}\SpecialCharTok{{-}}\DecValTok{1}\NormalTok{, ]}


\CommentTok{\# 2020 {-}{-}{-}{-}{-}{-}{-}{-}{-}{-}{-}{-}{-}{-}{-}{-}{-}{-}{-}{-}{-}{-}{-}{-}{-}{-}{-}{-}{-}{-}{-}{-}{-}{-}{-}{-}{-}{-}{-}{-}{-}{-}{-}{-}{-}{-}{-}{-}{-}{-}{-}{-}{-}{-}{-}{-}{-}{-}{-}{-}{-}{-}{-}{-}{-}{-}{-}{-}}

\NormalTok{tablas2020general }\OtherTok{\textless{}{-}} \FunctionTok{read.csv}\NormalTok{(}\StringTok{"2020.csv"}\NormalTok{, }\AttributeTok{sep=}\StringTok{"}\SpecialCharTok{\textbackslash{}t}\StringTok{"}\NormalTok{, }\AttributeTok{header=}\ConstantTok{TRUE}\NormalTok{, }\AttributeTok{fileEncoding=}\StringTok{"UTF{-}16LE"}\NormalTok{)}
\NormalTok{df17 }\OtherTok{\textless{}{-}}\NormalTok{ tablas2020general[}\SpecialCharTok{{-}}\DecValTok{1}\NormalTok{, ]}


\CommentTok{\# 2021 {-}{-}{-}{-}{-}{-}{-}{-}{-}{-}{-}{-}{-}{-}{-}{-}{-}{-}{-}{-}{-}{-}{-}{-}{-}{-}{-}{-}{-}{-}{-}{-}{-}{-}{-}{-}{-}{-}{-}{-}{-}{-}{-}{-}{-}{-}{-}{-}{-}{-}{-}{-}{-}{-}{-}{-}{-}{-}{-}{-}{-}{-}{-}{-}{-}{-}{-}{-}}

\NormalTok{tablas2021general }\OtherTok{\textless{}{-}} \FunctionTok{read.csv}\NormalTok{(}\StringTok{"2021.csv"}\NormalTok{, }\AttributeTok{sep=}\StringTok{"}\SpecialCharTok{\textbackslash{}t}\StringTok{"}\NormalTok{, }\AttributeTok{header=}\ConstantTok{TRUE}\NormalTok{, }\AttributeTok{fileEncoding=}\StringTok{"UTF{-}16LE"}\NormalTok{)}
\NormalTok{df18 }\OtherTok{\textless{}{-}}\NormalTok{ tablas2021general[}\SpecialCharTok{{-}}\DecValTok{1}\NormalTok{, ]}
\end{Highlighting}
\end{Shaded}

Procederemos ahora a seleccionar únicamente aquellas variables que
coincidan con las presentes en el data frame correspondiente a los datos
de 1895 a 2005.

\begin{Shaded}
\begin{Highlighting}[]
\NormalTok{datos\_lc2006 }\OtherTok{\textless{}{-}} \FunctionTok{select}\NormalTok{(df3,  year, age, mx, qx, lx, Lx, Tx, ex )}
\NormalTok{datos\_lc2007 }\OtherTok{\textless{}{-}} \FunctionTok{select}\NormalTok{(df4,  year, age, mx, qx, lx, Lx, Tx, ex )}
\NormalTok{datos\_lc2008 }\OtherTok{\textless{}{-}} \FunctionTok{select}\NormalTok{(df5,  year, age, mx, qx, lx, Lx, Tx, ex )}
\NormalTok{datos\_lc2009 }\OtherTok{\textless{}{-}} \FunctionTok{select}\NormalTok{(df6, year, age, mx, qx, lx, Lx, Tx, ex )}
\NormalTok{datos\_lc2010 }\OtherTok{\textless{}{-}} \FunctionTok{select}\NormalTok{(df7, year, age, mx, qx, lx, Lx, Tx, ex )}
\NormalTok{datos\_lc2011 }\OtherTok{\textless{}{-}} \FunctionTok{select}\NormalTok{(df8, year, age, mx, qx, lx, Lx, Tx, ex )}
\NormalTok{datos\_lc2012 }\OtherTok{\textless{}{-}} \FunctionTok{select}\NormalTok{(df9, year, age, mx, qx, lx, Lx, Tx, ex )}
\NormalTok{datos\_lc2013 }\OtherTok{\textless{}{-}} \FunctionTok{select}\NormalTok{(df10, year, age, mx, qx, lx, Lx, Tx, ex )}
\NormalTok{datos\_lc2014 }\OtherTok{\textless{}{-}} \FunctionTok{select}\NormalTok{(df11, year, age, mx, qx, lx, Lx, Tx, ex )}
\NormalTok{datos\_lc2015 }\OtherTok{\textless{}{-}} \FunctionTok{select}\NormalTok{(df12, year, age, mx, qx, lx, Lx, Tx, ex )}
\NormalTok{datos\_lc2016 }\OtherTok{\textless{}{-}} \FunctionTok{select}\NormalTok{(df13, year, age, mx, qx, lx, Lx, Tx, ex )}
\NormalTok{datos\_lc2017 }\OtherTok{\textless{}{-}} \FunctionTok{select}\NormalTok{(df14, year, age, mx, qx, lx, Lx, Tx, ex )}
\NormalTok{datos\_lc2018 }\OtherTok{\textless{}{-}} \FunctionTok{select}\NormalTok{(df15, year, age, mx, qx, lx, Lx, Tx, ex )}
\NormalTok{datos\_lc2019 }\OtherTok{\textless{}{-}} \FunctionTok{select}\NormalTok{(df16, year, age, mx, qx, lx, Lx, Tx, ex )}
\NormalTok{datos\_lc2020 }\OtherTok{\textless{}{-}} \FunctionTok{select}\NormalTok{(df17, year, age, mx, qx, lx, Lx, Tx, ex )}
\NormalTok{datos\_lc2021 }\OtherTok{\textless{}{-}} \FunctionTok{select}\NormalTok{(df18, year, age, mx, qx, lx, Lx, Tx, ex )}
\end{Highlighting}
\end{Shaded}

Realizamos el merge con el objetivo de integrar las bases de datos
correspondientes a periodos posteriores. Es importante señalar que esta
fusión se llevará a cabo de manera progresiva, combinando los conjuntos
de datos de forma secuencial y por pares.

\begin{Shaded}
\begin{Highlighting}[]
\NormalTok{merge1 }\OtherTok{\textless{}{-}} \FunctionTok{merge}\NormalTok{(datos\_lc2005, datos\_lc2006, }\AttributeTok{by=}\FunctionTok{c}\NormalTok{(}\StringTok{"year"}\NormalTok{,}\StringTok{"age"}\NormalTok{, }\StringTok{"mx"}\NormalTok{, }\StringTok{"qx"}\NormalTok{, }\StringTok{"lx"}\NormalTok{, }\StringTok{"Lx"}\NormalTok{, }\StringTok{"Tx"}\NormalTok{, }\StringTok{"ex"}\NormalTok{), }\AttributeTok{all.x =} \ConstantTok{TRUE}\NormalTok{, }\AttributeTok{all.y =} \ConstantTok{TRUE}\NormalTok{)}

\NormalTok{merge2 }\OtherTok{\textless{}{-}} \FunctionTok{merge}\NormalTok{(merge1, datos\_lc2007, }\AttributeTok{by=}\FunctionTok{c}\NormalTok{(}\StringTok{"year"}\NormalTok{,}\StringTok{"age"}\NormalTok{, }\StringTok{"mx"}\NormalTok{, }\StringTok{"qx"}\NormalTok{, }\StringTok{"lx"}\NormalTok{, }\StringTok{"Lx"}\NormalTok{, }\StringTok{"Tx"}\NormalTok{, }\StringTok{"ex"}\NormalTok{), }\AttributeTok{all.x =} \ConstantTok{TRUE}\NormalTok{, }\AttributeTok{all.y =} \ConstantTok{TRUE}\NormalTok{)}

\NormalTok{merge3 }\OtherTok{\textless{}{-}} \FunctionTok{merge}\NormalTok{(merge2, datos\_lc2008, }\AttributeTok{by=}\FunctionTok{c}\NormalTok{(}\StringTok{"year"}\NormalTok{,}\StringTok{"age"}\NormalTok{, }\StringTok{"mx"}\NormalTok{, }\StringTok{"qx"}\NormalTok{, }\StringTok{"lx"}\NormalTok{, }\StringTok{"Lx"}\NormalTok{, }\StringTok{"Tx"}\NormalTok{, }\StringTok{"ex"}\NormalTok{), }\AttributeTok{all.x =} \ConstantTok{TRUE}\NormalTok{, }\AttributeTok{all.y =} \ConstantTok{TRUE}\NormalTok{)}

\NormalTok{merge4 }\OtherTok{\textless{}{-}} \FunctionTok{merge}\NormalTok{(merge3, datos\_lc2009, }\AttributeTok{by=}\FunctionTok{c}\NormalTok{(}\StringTok{"year"}\NormalTok{,}\StringTok{"age"}\NormalTok{, }\StringTok{"mx"}\NormalTok{, }\StringTok{"qx"}\NormalTok{, }\StringTok{"lx"}\NormalTok{, }\StringTok{"Lx"}\NormalTok{, }\StringTok{"Tx"}\NormalTok{, }\StringTok{"ex"}\NormalTok{), }\AttributeTok{all.x =} \ConstantTok{TRUE}\NormalTok{, }\AttributeTok{all.y =} \ConstantTok{TRUE}\NormalTok{)}

\NormalTok{merge5 }\OtherTok{\textless{}{-}} \FunctionTok{merge}\NormalTok{(merge4, datos\_lc2010, }\AttributeTok{by=}\FunctionTok{c}\NormalTok{(}\StringTok{"year"}\NormalTok{,}\StringTok{"age"}\NormalTok{, }\StringTok{"mx"}\NormalTok{, }\StringTok{"qx"}\NormalTok{, }\StringTok{"lx"}\NormalTok{, }\StringTok{"Lx"}\NormalTok{, }\StringTok{"Tx"}\NormalTok{, }\StringTok{"ex"}\NormalTok{), }\AttributeTok{all.x =} \ConstantTok{TRUE}\NormalTok{, }\AttributeTok{all.y =} \ConstantTok{TRUE}\NormalTok{)}

\NormalTok{merge6 }\OtherTok{\textless{}{-}} \FunctionTok{merge}\NormalTok{(merge5, datos\_lc2011, }\AttributeTok{by=}\FunctionTok{c}\NormalTok{(}\StringTok{"year"}\NormalTok{,}\StringTok{"age"}\NormalTok{, }\StringTok{"mx"}\NormalTok{, }\StringTok{"qx"}\NormalTok{, }\StringTok{"lx"}\NormalTok{, }\StringTok{"Lx"}\NormalTok{, }\StringTok{"Tx"}\NormalTok{, }\StringTok{"ex"}\NormalTok{), }\AttributeTok{all.x =} \ConstantTok{TRUE}\NormalTok{, }\AttributeTok{all.y =} \ConstantTok{TRUE}\NormalTok{)}

\NormalTok{merge7 }\OtherTok{\textless{}{-}} \FunctionTok{merge}\NormalTok{(merge6, datos\_lc2012, }\AttributeTok{by=}\FunctionTok{c}\NormalTok{(}\StringTok{"year"}\NormalTok{,}\StringTok{"age"}\NormalTok{, }\StringTok{"mx"}\NormalTok{, }\StringTok{"qx"}\NormalTok{, }\StringTok{"lx"}\NormalTok{, }\StringTok{"Lx"}\NormalTok{, }\StringTok{"Tx"}\NormalTok{, }\StringTok{"ex"}\NormalTok{), }\AttributeTok{all.x =} \ConstantTok{TRUE}\NormalTok{, }\AttributeTok{all.y =} \ConstantTok{TRUE}\NormalTok{)}

\NormalTok{merge8 }\OtherTok{\textless{}{-}} \FunctionTok{merge}\NormalTok{(merge7, datos\_lc2013, }\AttributeTok{by=}\FunctionTok{c}\NormalTok{(}\StringTok{"year"}\NormalTok{,}\StringTok{"age"}\NormalTok{, }\StringTok{"mx"}\NormalTok{, }\StringTok{"qx"}\NormalTok{, }\StringTok{"lx"}\NormalTok{, }\StringTok{"Lx"}\NormalTok{, }\StringTok{"Tx"}\NormalTok{, }\StringTok{"ex"}\NormalTok{), }\AttributeTok{all.x =} \ConstantTok{TRUE}\NormalTok{, }\AttributeTok{all.y =} \ConstantTok{TRUE}\NormalTok{)}

\NormalTok{merge9 }\OtherTok{\textless{}{-}} \FunctionTok{merge}\NormalTok{(merge8, datos\_lc2014, }\AttributeTok{by=}\FunctionTok{c}\NormalTok{(}\StringTok{"year"}\NormalTok{,}\StringTok{"age"}\NormalTok{, }\StringTok{"mx"}\NormalTok{, }\StringTok{"qx"}\NormalTok{, }\StringTok{"lx"}\NormalTok{, }\StringTok{"Lx"}\NormalTok{, }\StringTok{"Tx"}\NormalTok{, }\StringTok{"ex"}\NormalTok{), }\AttributeTok{all.x =} \ConstantTok{TRUE}\NormalTok{, }\AttributeTok{all.y =} \ConstantTok{TRUE}\NormalTok{)}

\NormalTok{merge10 }\OtherTok{\textless{}{-}} \FunctionTok{merge}\NormalTok{(merge9, datos\_lc2015, }\AttributeTok{by=}\FunctionTok{c}\NormalTok{(}\StringTok{"year"}\NormalTok{,}\StringTok{"age"}\NormalTok{, }\StringTok{"mx"}\NormalTok{, }\StringTok{"qx"}\NormalTok{, }\StringTok{"lx"}\NormalTok{, }\StringTok{"Lx"}\NormalTok{, }\StringTok{"Tx"}\NormalTok{, }\StringTok{"ex"}\NormalTok{), }\AttributeTok{all.x =} \ConstantTok{TRUE}\NormalTok{, }\AttributeTok{all.y =} \ConstantTok{TRUE}\NormalTok{)}

\NormalTok{merge11 }\OtherTok{\textless{}{-}} \FunctionTok{merge}\NormalTok{(merge10, datos\_lc2016, }\AttributeTok{by=}\FunctionTok{c}\NormalTok{(}\StringTok{"year"}\NormalTok{,}\StringTok{"age"}\NormalTok{, }\StringTok{"mx"}\NormalTok{, }\StringTok{"qx"}\NormalTok{, }\StringTok{"lx"}\NormalTok{, }\StringTok{"Lx"}\NormalTok{, }\StringTok{"Tx"}\NormalTok{, }\StringTok{"ex"}\NormalTok{), }\AttributeTok{all.x =} \ConstantTok{TRUE}\NormalTok{, }\AttributeTok{all.y =} \ConstantTok{TRUE}\NormalTok{)}

\NormalTok{merge12 }\OtherTok{\textless{}{-}} \FunctionTok{merge}\NormalTok{(merge11, datos\_lc2017, }\AttributeTok{by=}\FunctionTok{c}\NormalTok{(}\StringTok{"year"}\NormalTok{,}\StringTok{"age"}\NormalTok{, }\StringTok{"mx"}\NormalTok{, }\StringTok{"qx"}\NormalTok{, }\StringTok{"lx"}\NormalTok{, }\StringTok{"Lx"}\NormalTok{, }\StringTok{"Tx"}\NormalTok{, }\StringTok{"ex"}\NormalTok{), }\AttributeTok{all.x =} \ConstantTok{TRUE}\NormalTok{, }\AttributeTok{all.y =} \ConstantTok{TRUE}\NormalTok{)}

\NormalTok{merge13 }\OtherTok{\textless{}{-}} \FunctionTok{merge}\NormalTok{(merge12, datos\_lc2018, }\AttributeTok{by=}\FunctionTok{c}\NormalTok{(}\StringTok{"year"}\NormalTok{,}\StringTok{"age"}\NormalTok{, }\StringTok{"mx"}\NormalTok{, }\StringTok{"qx"}\NormalTok{, }\StringTok{"lx"}\NormalTok{, }\StringTok{"Lx"}\NormalTok{, }\StringTok{"Tx"}\NormalTok{, }\StringTok{"ex"}\NormalTok{), }\AttributeTok{all.x =} \ConstantTok{TRUE}\NormalTok{, }\AttributeTok{all.y =} \ConstantTok{TRUE}\NormalTok{)}

\NormalTok{merge14 }\OtherTok{\textless{}{-}} \FunctionTok{merge}\NormalTok{(merge13, datos\_lc2019, }\AttributeTok{by=}\FunctionTok{c}\NormalTok{(}\StringTok{"year"}\NormalTok{,}\StringTok{"age"}\NormalTok{, }\StringTok{"mx"}\NormalTok{, }\StringTok{"qx"}\NormalTok{, }\StringTok{"lx"}\NormalTok{, }\StringTok{"Lx"}\NormalTok{, }\StringTok{"Tx"}\NormalTok{, }\StringTok{"ex"}\NormalTok{), }\AttributeTok{all.x =} \ConstantTok{TRUE}\NormalTok{, }\AttributeTok{all.y =} \ConstantTok{TRUE}\NormalTok{)}

\NormalTok{merge15 }\OtherTok{\textless{}{-}} \FunctionTok{merge}\NormalTok{(merge14, datos\_lc2020, }\AttributeTok{by=}\FunctionTok{c}\NormalTok{(}\StringTok{"year"}\NormalTok{,}\StringTok{"age"}\NormalTok{, }\StringTok{"mx"}\NormalTok{, }\StringTok{"qx"}\NormalTok{, }\StringTok{"lx"}\NormalTok{, }\StringTok{"Lx"}\NormalTok{, }\StringTok{"Tx"}\NormalTok{, }\StringTok{"ex"}\NormalTok{), }\AttributeTok{all.x =} \ConstantTok{TRUE}\NormalTok{, }\AttributeTok{all.y =} \ConstantTok{TRUE}\NormalTok{)}

\NormalTok{merge16 }\OtherTok{\textless{}{-}} \FunctionTok{merge}\NormalTok{(merge15, datos\_lc2021, }\AttributeTok{by=}\FunctionTok{c}\NormalTok{(}\StringTok{"year"}\NormalTok{,}\StringTok{"age"}\NormalTok{, }\StringTok{"mx"}\NormalTok{, }\StringTok{"qx"}\NormalTok{, }\StringTok{"lx"}\NormalTok{, }\StringTok{"Lx"}\NormalTok{, }\StringTok{"Tx"}\NormalTok{, }\StringTok{"ex"}\NormalTok{), }\AttributeTok{all.x =} \ConstantTok{TRUE}\NormalTok{, }\AttributeTok{all.y =} \ConstantTok{TRUE}\NormalTok{)}
\end{Highlighting}
\end{Shaded}

Definimos la base final ``base\_final\_hombres'', la cual ya integra los
datos correspondientes al periodo comprendido entre 1895 y 2021.

\begin{Shaded}
\begin{Highlighting}[]
\NormalTok{base\_final\_hombres }\OtherTok{\textless{}{-}}\NormalTok{ merge16}
\end{Highlighting}
\end{Shaded}

Durante el proceso de revisión y exploración de los datos, detectamos
que algunos registros presentaban comas en lugares inesperados, lo que
alteraba el formato numérico adecuado. Este inconveniente afectaba el
procesamiento correcto de la información, pudiendo distorsionar nuestros
cálculos y análisis posteriores.

\begin{Shaded}
\begin{Highlighting}[]
\CommentTok{\# Verificacion de los datos }
\NormalTok{columnas\_con\_coma }\OtherTok{\textless{}{-}} \FunctionTok{sapply}\NormalTok{(base\_final\_hombres, }\ControlFlowTok{function}\NormalTok{(x) }\FunctionTok{any}\NormalTok{(}\FunctionTok{grepl}\NormalTok{(}\StringTok{","}\NormalTok{,x)))}
\FunctionTok{print}\NormalTok{(columnas\_con\_coma)}
\end{Highlighting}
\end{Shaded}

\begin{verbatim}
##  year   age    mx    qx    lx    Lx    Tx    ex 
## FALSE FALSE FALSE FALSE  TRUE  TRUE  TRUE FALSE
\end{verbatim}

\begin{Shaded}
\begin{Highlighting}[]
\CommentTok{\# Eliminamos las comas}
\NormalTok{base\_final\_hombres}\SpecialCharTok{$}\NormalTok{lx }\OtherTok{\textless{}{-}} \FunctionTok{gsub}\NormalTok{(}\StringTok{","}\NormalTok{,}\StringTok{""}\NormalTok{,base\_final\_hombres}\SpecialCharTok{$}\NormalTok{lx)}

\CommentTok{\# Detectamos la presencia de mas comas en columnas Tx y ex}
\CommentTok{\# Columna Tx}
\NormalTok{base\_final\_hombres}\SpecialCharTok{$}\NormalTok{Tx }\OtherTok{\textless{}{-}} \FunctionTok{gsub}\NormalTok{(}\StringTok{","}\NormalTok{,}\StringTok{""}\NormalTok{,base\_final\_hombres}\SpecialCharTok{$}\NormalTok{Tx)}

\CommentTok{\# Columna ex}
\NormalTok{base\_final\_hombres}\SpecialCharTok{$}\NormalTok{Lx }\OtherTok{\textless{}{-}} \FunctionTok{gsub}\NormalTok{(}\StringTok{","}\NormalTok{,}\StringTok{""}\NormalTok{,base\_final\_hombres}\SpecialCharTok{$}\NormalTok{Lx)}
\end{Highlighting}
\end{Shaded}

Una vez concluido el proceso de limpieza de comas, procederemos a
examinar detenidamente la estructura de la base de datos, verificando
que cada uno de sus componentes se encuentre organizado según lo
esperado. Posteriormente, convertiremos todas nuestras variables al tipo
numérico

\begin{Shaded}
\begin{Highlighting}[]
\NormalTok{base\_final\_hombres}\SpecialCharTok{$}\NormalTok{year }\OtherTok{\textless{}{-}} \FunctionTok{as.numeric}\NormalTok{(base\_final\_hombres}\SpecialCharTok{$}\NormalTok{year)}
\NormalTok{base\_final\_hombres}\SpecialCharTok{$}\NormalTok{age }\OtherTok{\textless{}{-}} \FunctionTok{as.numeric}\NormalTok{(base\_final\_hombres}\SpecialCharTok{$}\NormalTok{age)}
\NormalTok{base\_final\_hombres}\SpecialCharTok{$}\NormalTok{mx }\OtherTok{\textless{}{-}} \FunctionTok{as.numeric}\NormalTok{(base\_final\_hombres}\SpecialCharTok{$}\NormalTok{mx)}
\NormalTok{base\_final\_hombres}\SpecialCharTok{$}\NormalTok{qx }\OtherTok{\textless{}{-}} \FunctionTok{as.numeric}\NormalTok{(base\_final\_hombres}\SpecialCharTok{$}\NormalTok{qx)}
\NormalTok{base\_final\_hombres}\SpecialCharTok{$}\NormalTok{lx }\OtherTok{\textless{}{-}} \FunctionTok{as.numeric}\NormalTok{(base\_final\_hombres}\SpecialCharTok{$}\NormalTok{lx)}
\NormalTok{base\_final\_hombres}\SpecialCharTok{$}\NormalTok{Lx }\OtherTok{\textless{}{-}} \FunctionTok{as.numeric}\NormalTok{(base\_final\_hombres}\SpecialCharTok{$}\NormalTok{Lx)}
\NormalTok{base\_final\_hombres}\SpecialCharTok{$}\NormalTok{Tx }\OtherTok{\textless{}{-}} \FunctionTok{as.numeric}\NormalTok{(base\_final\_hombres}\SpecialCharTok{$}\NormalTok{Tx)}
\NormalTok{base\_final\_hombres}\SpecialCharTok{$}\NormalTok{ex }\OtherTok{\textless{}{-}} \FunctionTok{as.numeric}\NormalTok{(base\_final\_hombres}\SpecialCharTok{$}\NormalTok{ex)}
\end{Highlighting}
\end{Shaded}

Procedamos a realizar un análisis de correlación lineal para investigar
si se establecen relaciones significativas entre las variables de
nuestro conjunto de datos.

\begin{Shaded}
\begin{Highlighting}[]
\FunctionTok{cor}\NormalTok{(base\_final\_hombres[, }\FunctionTok{c}\NormalTok{(}\StringTok{"year"}\NormalTok{,}\StringTok{"age"}\NormalTok{,}\StringTok{"mx"}\NormalTok{,}\StringTok{"qx"}\NormalTok{,}\StringTok{"lx"}\NormalTok{,}\StringTok{"Lx"}\NormalTok{,}\StringTok{"Tx"}\NormalTok{,}\StringTok{"ex"}\NormalTok{)])}
\end{Highlighting}
\end{Shaded}

\begin{verbatim}
##               year           age         mx         qx         lx         Lx
## year  1.000000e+00 -5.232687e-20 -0.3927207 -0.2375079  0.6373840  0.6345414
## age  -5.232687e-20  1.000000e+00  0.3934822  0.6774502 -0.6918712 -0.4574899
## mx   -3.927207e-01  3.934822e-01  1.0000000  0.8202217 -0.5564214 -0.6771687
## qx   -2.375079e-01  6.774502e-01  0.8202217  1.0000000 -0.7457807 -0.6875409
## lx    6.373840e-01 -6.918712e-01 -0.5564214 -0.7457807  1.0000000  0.8236354
## Lx    6.345414e-01 -4.574899e-01 -0.6771687 -0.6875409  0.8236354  1.0000000
## Tx    4.667957e-01 -8.272127e-01 -0.4266558 -0.5874758  0.8411394  0.6327994
## ex    3.281810e-01 -9.104004e-01 -0.4947681 -0.6558508  0.8004225  0.6162034
##              Tx         ex
## year  0.4667957  0.3281810
## age  -0.8272127 -0.9104004
## mx   -0.4266558 -0.4947681
## qx   -0.5874758 -0.6558508
## lx    0.8411394  0.8004225
## Lx    0.6327994  0.6162034
## Tx    1.0000000  0.9749885
## ex    0.9749885  1.0000000
\end{verbatim}

Podemos observar que existen relaciones signifativas. Por ejemplo entre
\emph{mx} y \emph{qx}, \emph{lx} y \emph{Lx}, \emph{Tx} y \emph{lx},
\emph{Tx} y \emph{Lx} , \emph{Tx} y \emph{ex}, entre otros.

Para el caso de las mujeres, se aplicó el mismo procedimiento. Con el
fin de mantener la brevedad y evitar extendernos innecesariamente, a
continuación se presenta únicamente el código utilizado junto con sus
respectivos resultados.

\begin{Shaded}
\begin{Highlighting}[]
\CommentTok{\# LIMPIEZA DE LA BASE DE DATOS}
\CommentTok{\# Leemos los de intervalos de 5}
\NormalTok{datos\_lc5\_m }\OtherTok{\textless{}{-}} \FunctionTok{read.table}\NormalTok{(}\StringTok{"Mex\_fltpiv\_5x1\_mujeres.txt"}\NormalTok{, }\AttributeTok{header =}\NormalTok{ T, }\AttributeTok{sep =} \StringTok{""}\NormalTok{)}

\CommentTok{\# Nos quedamos con las variables de nuestro interes}
\NormalTok{datos\_lc2005\_m }\OtherTok{\textless{}{-}} \FunctionTok{select}\NormalTok{(datos\_lc5\_m, year, age, mxm, qxm, lxm, Lxm, Txm, exm)}

\CommentTok{\# NOTA:}
\CommentTok{\# Hasta aqui podemos ver que tenemos los datos desde 1895{-}2005}
\CommentTok{\# Continuamos con las demas tablas}
\NormalTok{datos\_lc2006\_m }\OtherTok{\textless{}{-}} \FunctionTok{select}\NormalTok{(df3,  year, age, mxm, qxm, lxm, Lxm, Txm, exm )}
\NormalTok{datos\_lc2007\_m }\OtherTok{\textless{}{-}} \FunctionTok{select}\NormalTok{(df4,  year, age, mxm, qxm, lxm, Lxm, Txm, exm )}
\NormalTok{datos\_lc2008\_m }\OtherTok{\textless{}{-}} \FunctionTok{select}\NormalTok{(df5,  year, age, mxm, qxm, lxm, Lxm, Txm, exm )}
\NormalTok{datos\_lc2009\_m }\OtherTok{\textless{}{-}} \FunctionTok{select}\NormalTok{(df6, year, age, mxm, qxm, lxm, Lxm, Txm, exm )}
\NormalTok{datos\_lc2010\_m }\OtherTok{\textless{}{-}} \FunctionTok{select}\NormalTok{(df7, year, age, mxm, qxm, lxm, Lxm, Txm, exm )}
\NormalTok{datos\_lc2011\_m }\OtherTok{\textless{}{-}} \FunctionTok{select}\NormalTok{(df8, year, age, mxm, qxm, lxm, Lxm, Txm, exm )}
\NormalTok{datos\_lc2012\_m }\OtherTok{\textless{}{-}} \FunctionTok{select}\NormalTok{(df9, year, age, mxm, qxm, lxm, Lxm, Txm, exm )}
\NormalTok{datos\_lc2013\_m }\OtherTok{\textless{}{-}} \FunctionTok{select}\NormalTok{(df10, year, age, mxm, qxm, lxm, Lxm, Txm, exm )}
\NormalTok{datos\_lc2014\_m }\OtherTok{\textless{}{-}} \FunctionTok{select}\NormalTok{(df11, year, age, mxm, qxm, lxm, Lxm, Txm, exm )}
\NormalTok{datos\_lc2015\_m }\OtherTok{\textless{}{-}} \FunctionTok{select}\NormalTok{(df12, year, age, mxm, qxm, lxm, Lxm, Txm, exm )}
\NormalTok{datos\_lc2016\_m }\OtherTok{\textless{}{-}} \FunctionTok{select}\NormalTok{(df13, year, age, mxm, qxm, lxm, Lxm, Txm, exm )}
\NormalTok{datos\_lc2017\_m }\OtherTok{\textless{}{-}} \FunctionTok{select}\NormalTok{(df14, year, age, mxm, qxm, lxm, Lxm, Txm, exm )}
\NormalTok{datos\_lc2018\_m }\OtherTok{\textless{}{-}} \FunctionTok{select}\NormalTok{(df15, year, age, mxm, qxm, lxm, Lxm, Txm, exm )}
\NormalTok{datos\_lc2019\_m }\OtherTok{\textless{}{-}} \FunctionTok{select}\NormalTok{(df16, year, age, mxm, qxm, lxm, Lxm, Txm, exm )}
\NormalTok{datos\_lc2020\_m }\OtherTok{\textless{}{-}} \FunctionTok{select}\NormalTok{(df17, year, age, mxm, qxm, lxm, Lxm, Txm, exm )}
\NormalTok{datos\_lc2021\_m }\OtherTok{\textless{}{-}} \FunctionTok{select}\NormalTok{(df18, year, age, mxm, qxm, lxm, Lxm, Txm, exm )}


\CommentTok{\# Realizamos el merge}
\CommentTok{\# Esto es con la finalidad de ir juntando las bases de datos de periodos posteriores}
\CommentTok{\# Debemos de ir haciendolo por pares}
\NormalTok{merge1m }\OtherTok{\textless{}{-}} \FunctionTok{merge}\NormalTok{(datos\_lc2005\_m, datos\_lc2006\_m, }\AttributeTok{by=}\FunctionTok{c}\NormalTok{(}\StringTok{"year"}\NormalTok{,}\StringTok{"age"}\NormalTok{, }\StringTok{"mxm"}\NormalTok{, }\StringTok{"qxm"}\NormalTok{, }\StringTok{"lxm"}\NormalTok{, }\StringTok{"Lxm"}\NormalTok{, }\StringTok{"Txm"}\NormalTok{, }\StringTok{"exm"}\NormalTok{), }\AttributeTok{all.x =} \ConstantTok{TRUE}\NormalTok{, }\AttributeTok{all.y =} \ConstantTok{TRUE}\NormalTok{)}

\NormalTok{merge2m }\OtherTok{\textless{}{-}} \FunctionTok{merge}\NormalTok{(merge1m, datos\_lc2007\_m, }\AttributeTok{by=}\FunctionTok{c}\NormalTok{(}\StringTok{"year"}\NormalTok{,}\StringTok{"age"}\NormalTok{, }\StringTok{"mxm"}\NormalTok{, }\StringTok{"qxm"}\NormalTok{, }\StringTok{"lxm"}\NormalTok{, }\StringTok{"Lxm"}\NormalTok{, }\StringTok{"Txm"}\NormalTok{, }\StringTok{"exm"}\NormalTok{), }\AttributeTok{all.x =} \ConstantTok{TRUE}\NormalTok{, }\AttributeTok{all.y =} \ConstantTok{TRUE}\NormalTok{)}

\NormalTok{merge3m }\OtherTok{\textless{}{-}} \FunctionTok{merge}\NormalTok{(merge2m, datos\_lc2008\_m, }\AttributeTok{by=}\FunctionTok{c}\NormalTok{(}\StringTok{"year"}\NormalTok{,}\StringTok{"age"}\NormalTok{, }\StringTok{"mxm"}\NormalTok{, }\StringTok{"qxm"}\NormalTok{, }\StringTok{"lxm"}\NormalTok{, }\StringTok{"Lxm"}\NormalTok{, }\StringTok{"Txm"}\NormalTok{, }\StringTok{"exm"}\NormalTok{), }\AttributeTok{all.x =} \ConstantTok{TRUE}\NormalTok{, }\AttributeTok{all.y =} \ConstantTok{TRUE}\NormalTok{)}

\NormalTok{merge4m }\OtherTok{\textless{}{-}} \FunctionTok{merge}\NormalTok{(merge3m, datos\_lc2009\_m, }\AttributeTok{by=}\FunctionTok{c}\NormalTok{(}\StringTok{"year"}\NormalTok{,}\StringTok{"age"}\NormalTok{, }\StringTok{"mxm"}\NormalTok{, }\StringTok{"qxm"}\NormalTok{, }\StringTok{"lxm"}\NormalTok{, }\StringTok{"Lxm"}\NormalTok{, }\StringTok{"Txm"}\NormalTok{, }\StringTok{"exm"}\NormalTok{), }\AttributeTok{all.x =} \ConstantTok{TRUE}\NormalTok{, }\AttributeTok{all.y =} \ConstantTok{TRUE}\NormalTok{)}

\NormalTok{merge5m }\OtherTok{\textless{}{-}} \FunctionTok{merge}\NormalTok{(merge4m, datos\_lc2010\_m, }\AttributeTok{by=}\FunctionTok{c}\NormalTok{(}\StringTok{"year"}\NormalTok{,}\StringTok{"age"}\NormalTok{, }\StringTok{"mxm"}\NormalTok{, }\StringTok{"qxm"}\NormalTok{, }\StringTok{"lxm"}\NormalTok{, }\StringTok{"Lxm"}\NormalTok{, }\StringTok{"Txm"}\NormalTok{, }\StringTok{"exm"}\NormalTok{), }\AttributeTok{all.x =} \ConstantTok{TRUE}\NormalTok{, }\AttributeTok{all.y =} \ConstantTok{TRUE}\NormalTok{)}

\NormalTok{merge6m }\OtherTok{\textless{}{-}} \FunctionTok{merge}\NormalTok{(merge5m, datos\_lc2011\_m, }\AttributeTok{by=}\FunctionTok{c}\NormalTok{(}\StringTok{"year"}\NormalTok{,}\StringTok{"age"}\NormalTok{, }\StringTok{"mxm"}\NormalTok{, }\StringTok{"qxm"}\NormalTok{, }\StringTok{"lxm"}\NormalTok{, }\StringTok{"Lxm"}\NormalTok{, }\StringTok{"Txm"}\NormalTok{, }\StringTok{"exm"}\NormalTok{), }\AttributeTok{all.x =} \ConstantTok{TRUE}\NormalTok{, }\AttributeTok{all.y =} \ConstantTok{TRUE}\NormalTok{)}

\NormalTok{merge7m }\OtherTok{\textless{}{-}} \FunctionTok{merge}\NormalTok{(merge6m, datos\_lc2012\_m, }\AttributeTok{by=}\FunctionTok{c}\NormalTok{(}\StringTok{"year"}\NormalTok{,}\StringTok{"age"}\NormalTok{, }\StringTok{"mxm"}\NormalTok{, }\StringTok{"qxm"}\NormalTok{, }\StringTok{"lxm"}\NormalTok{, }\StringTok{"Lxm"}\NormalTok{, }\StringTok{"Txm"}\NormalTok{, }\StringTok{"exm"}\NormalTok{), }\AttributeTok{all.x =} \ConstantTok{TRUE}\NormalTok{, }\AttributeTok{all.y =} \ConstantTok{TRUE}\NormalTok{)}

\NormalTok{merge8m }\OtherTok{\textless{}{-}} \FunctionTok{merge}\NormalTok{(merge7m, datos\_lc2013\_m, }\AttributeTok{by=}\FunctionTok{c}\NormalTok{(}\StringTok{"year"}\NormalTok{,}\StringTok{"age"}\NormalTok{, }\StringTok{"mxm"}\NormalTok{, }\StringTok{"qxm"}\NormalTok{, }\StringTok{"lxm"}\NormalTok{, }\StringTok{"Lxm"}\NormalTok{, }\StringTok{"Txm"}\NormalTok{, }\StringTok{"exm"}\NormalTok{), }\AttributeTok{all.x =} \ConstantTok{TRUE}\NormalTok{, }\AttributeTok{all.y =} \ConstantTok{TRUE}\NormalTok{)}

\NormalTok{merge9m }\OtherTok{\textless{}{-}} \FunctionTok{merge}\NormalTok{(merge8m, datos\_lc2014\_m, }\AttributeTok{by=}\FunctionTok{c}\NormalTok{(}\StringTok{"year"}\NormalTok{,}\StringTok{"age"}\NormalTok{, }\StringTok{"mxm"}\NormalTok{, }\StringTok{"qxm"}\NormalTok{, }\StringTok{"lxm"}\NormalTok{, }\StringTok{"Lxm"}\NormalTok{, }\StringTok{"Txm"}\NormalTok{, }\StringTok{"exm"}\NormalTok{), }\AttributeTok{all.x =} \ConstantTok{TRUE}\NormalTok{, }\AttributeTok{all.y =} \ConstantTok{TRUE}\NormalTok{)}

\NormalTok{merge10m }\OtherTok{\textless{}{-}} \FunctionTok{merge}\NormalTok{(merge9m, datos\_lc2015\_m, }\AttributeTok{by=}\FunctionTok{c}\NormalTok{(}\StringTok{"year"}\NormalTok{,}\StringTok{"age"}\NormalTok{, }\StringTok{"mxm"}\NormalTok{, }\StringTok{"qxm"}\NormalTok{, }\StringTok{"lxm"}\NormalTok{, }\StringTok{"Lxm"}\NormalTok{, }\StringTok{"Txm"}\NormalTok{, }\StringTok{"exm"}\NormalTok{), }\AttributeTok{all.x =} \ConstantTok{TRUE}\NormalTok{, }\AttributeTok{all.y =} \ConstantTok{TRUE}\NormalTok{)}

\NormalTok{merge11m }\OtherTok{\textless{}{-}} \FunctionTok{merge}\NormalTok{(merge10m, datos\_lc2016\_m, }\AttributeTok{by=}\FunctionTok{c}\NormalTok{(}\StringTok{"year"}\NormalTok{,}\StringTok{"age"}\NormalTok{, }\StringTok{"mxm"}\NormalTok{, }\StringTok{"qxm"}\NormalTok{, }\StringTok{"lxm"}\NormalTok{, }\StringTok{"Lxm"}\NormalTok{, }\StringTok{"Txm"}\NormalTok{, }\StringTok{"exm"}\NormalTok{), }\AttributeTok{all.x =} \ConstantTok{TRUE}\NormalTok{, }\AttributeTok{all.y =} \ConstantTok{TRUE}\NormalTok{)}

\NormalTok{merge12m }\OtherTok{\textless{}{-}} \FunctionTok{merge}\NormalTok{(merge11m, datos\_lc2017\_m, }\AttributeTok{by=}\FunctionTok{c}\NormalTok{(}\StringTok{"year"}\NormalTok{,}\StringTok{"age"}\NormalTok{, }\StringTok{"mxm"}\NormalTok{, }\StringTok{"qxm"}\NormalTok{, }\StringTok{"lxm"}\NormalTok{, }\StringTok{"Lxm"}\NormalTok{, }\StringTok{"Txm"}\NormalTok{, }\StringTok{"exm"}\NormalTok{), }\AttributeTok{all.x =} \ConstantTok{TRUE}\NormalTok{, }\AttributeTok{all.y =} \ConstantTok{TRUE}\NormalTok{)}

\NormalTok{merge13m }\OtherTok{\textless{}{-}} \FunctionTok{merge}\NormalTok{(merge12m, datos\_lc2018\_m, }\AttributeTok{by=}\FunctionTok{c}\NormalTok{(}\StringTok{"year"}\NormalTok{,}\StringTok{"age"}\NormalTok{, }\StringTok{"mxm"}\NormalTok{, }\StringTok{"qxm"}\NormalTok{, }\StringTok{"lxm"}\NormalTok{, }\StringTok{"Lxm"}\NormalTok{, }\StringTok{"Txm"}\NormalTok{, }\StringTok{"exm"}\NormalTok{), }\AttributeTok{all.x =} \ConstantTok{TRUE}\NormalTok{, }\AttributeTok{all.y =} \ConstantTok{TRUE}\NormalTok{)}

\NormalTok{merge14m }\OtherTok{\textless{}{-}} \FunctionTok{merge}\NormalTok{(merge13m, datos\_lc2019\_m, }\AttributeTok{by=}\FunctionTok{c}\NormalTok{(}\StringTok{"year"}\NormalTok{,}\StringTok{"age"}\NormalTok{, }\StringTok{"mxm"}\NormalTok{, }\StringTok{"qxm"}\NormalTok{, }\StringTok{"lxm"}\NormalTok{, }\StringTok{"Lxm"}\NormalTok{, }\StringTok{"Txm"}\NormalTok{, }\StringTok{"exm"}\NormalTok{), }\AttributeTok{all.x =} \ConstantTok{TRUE}\NormalTok{, }\AttributeTok{all.y =} \ConstantTok{TRUE}\NormalTok{)}

\NormalTok{merge15m }\OtherTok{\textless{}{-}} \FunctionTok{merge}\NormalTok{(merge14m, datos\_lc2020\_m, }\AttributeTok{by=}\FunctionTok{c}\NormalTok{(}\StringTok{"year"}\NormalTok{,}\StringTok{"age"}\NormalTok{, }\StringTok{"mxm"}\NormalTok{, }\StringTok{"qxm"}\NormalTok{, }\StringTok{"lxm"}\NormalTok{, }\StringTok{"Lxm"}\NormalTok{, }\StringTok{"Txm"}\NormalTok{, }\StringTok{"exm"}\NormalTok{), }\AttributeTok{all.x =} \ConstantTok{TRUE}\NormalTok{, }\AttributeTok{all.y =} \ConstantTok{TRUE}\NormalTok{)}

\NormalTok{merge16m }\OtherTok{\textless{}{-}} \FunctionTok{merge}\NormalTok{(merge15m, datos\_lc2021\_m, }\AttributeTok{by=}\FunctionTok{c}\NormalTok{(}\StringTok{"year"}\NormalTok{,}\StringTok{"age"}\NormalTok{, }\StringTok{"mxm"}\NormalTok{, }\StringTok{"qxm"}\NormalTok{, }\StringTok{"lxm"}\NormalTok{, }\StringTok{"Lxm"}\NormalTok{, }\StringTok{"Txm"}\NormalTok{, }\StringTok{"exm"}\NormalTok{), }\AttributeTok{all.x =} \ConstantTok{TRUE}\NormalTok{, }\AttributeTok{all.y =} \ConstantTok{TRUE}\NormalTok{)}


\DocumentationTok{\#\#Nombramos base final hombres}
\NormalTok{base\_final\_mujeres }\OtherTok{\textless{}{-}}\NormalTok{ merge16m}

\CommentTok{\# Verificacion de los datos }
\NormalTok{columnas\_con\_coma }\OtherTok{\textless{}{-}} \FunctionTok{sapply}\NormalTok{(base\_final\_mujeres, }\ControlFlowTok{function}\NormalTok{(x) }\FunctionTok{any}\NormalTok{(}\FunctionTok{grepl}\NormalTok{(}\StringTok{","}\NormalTok{,x)))}
\FunctionTok{print}\NormalTok{(columnas\_con\_coma)}
\end{Highlighting}
\end{Shaded}

\begin{verbatim}
##  year   age   mxm   qxm   lxm   Lxm   Txm   exm 
## FALSE FALSE FALSE FALSE  TRUE  TRUE  TRUE FALSE
\end{verbatim}

\begin{Shaded}
\begin{Highlighting}[]
\CommentTok{\# Eliminamos las comas}
\NormalTok{base\_final\_mujeres}\SpecialCharTok{$}\NormalTok{lxm }\OtherTok{\textless{}{-}} \FunctionTok{gsub}\NormalTok{(}\StringTok{","}\NormalTok{,}\StringTok{""}\NormalTok{,base\_final\_mujeres}\SpecialCharTok{$}\NormalTok{lxm)}

\CommentTok{\# Detectamos la presencia de mas comas en columnas Tx y ex}
\CommentTok{\# Columna Tx}
\NormalTok{base\_final\_mujeres}\SpecialCharTok{$}\NormalTok{Txm }\OtherTok{\textless{}{-}} \FunctionTok{gsub}\NormalTok{(}\StringTok{","}\NormalTok{,}\StringTok{""}\NormalTok{,base\_final\_mujeres}\SpecialCharTok{$}\NormalTok{Txm)}

\CommentTok{\# Columna ex}
\NormalTok{base\_final\_mujeres}\SpecialCharTok{$}\NormalTok{Lxm }\OtherTok{\textless{}{-}} \FunctionTok{gsub}\NormalTok{(}\StringTok{","}\NormalTok{,}\StringTok{""}\NormalTok{,base\_final\_mujeres}\SpecialCharTok{$}\NormalTok{Lxm)}

\CommentTok{\#Vamos a pasar las variables a numericas}
\NormalTok{base\_final\_mujeres}\SpecialCharTok{$}\NormalTok{year }\OtherTok{\textless{}{-}} \FunctionTok{as.numeric}\NormalTok{(base\_final\_mujeres}\SpecialCharTok{$}\NormalTok{year)}
\NormalTok{base\_final\_mujeres}\SpecialCharTok{$}\NormalTok{age }\OtherTok{\textless{}{-}} \FunctionTok{as.numeric}\NormalTok{(base\_final\_mujeres}\SpecialCharTok{$}\NormalTok{age)}
\NormalTok{base\_final\_mujeres}\SpecialCharTok{$}\NormalTok{mxm }\OtherTok{\textless{}{-}} \FunctionTok{as.numeric}\NormalTok{(base\_final\_mujeres}\SpecialCharTok{$}\NormalTok{mxm)}
\NormalTok{base\_final\_mujeres}\SpecialCharTok{$}\NormalTok{qxm }\OtherTok{\textless{}{-}} \FunctionTok{as.numeric}\NormalTok{(base\_final\_mujeres}\SpecialCharTok{$}\NormalTok{qxm)}
\NormalTok{base\_final\_mujeres}\SpecialCharTok{$}\NormalTok{lxm }\OtherTok{\textless{}{-}} \FunctionTok{as.numeric}\NormalTok{(base\_final\_mujeres}\SpecialCharTok{$}\NormalTok{lxm)}
\NormalTok{base\_final\_mujeres}\SpecialCharTok{$}\NormalTok{Lxm }\OtherTok{\textless{}{-}} \FunctionTok{as.numeric}\NormalTok{(base\_final\_mujeres}\SpecialCharTok{$}\NormalTok{Lxm)}
\NormalTok{base\_final\_mujeres}\SpecialCharTok{$}\NormalTok{Txm }\OtherTok{\textless{}{-}} \FunctionTok{as.numeric}\NormalTok{(base\_final\_mujeres}\SpecialCharTok{$}\NormalTok{Txm)}
\NormalTok{base\_final\_mujeres}\SpecialCharTok{$}\NormalTok{exm }\OtherTok{\textless{}{-}} \FunctionTok{as.numeric}\NormalTok{(base\_final\_mujeres}\SpecialCharTok{$}\NormalTok{exm)}
\end{Highlighting}
\end{Shaded}

Para culminar el punto 4, presentamos de forma gráfica las tasas de
mortalidad en función del factor de edad y año, tanto para hombres como
para mujeres. Con este análisis visual, buscamos identificar patrones y
tendencias que reflejen la evolución de estas tasas a lo largo del
tiempo.

\begin{Shaded}
\begin{Highlighting}[]
\CommentTok{\# TASAS DE MORTALIDAD MUJERES {-}{-}{-}{-}{-}{-}{-}{-}{-}{-}{-}{-}{-}{-}{-}{-}{-}{-}{-}{-}{-}{-}{-}{-}{-}{-}{-}{-}{-}{-}{-}{-}{-}{-}{-}{-}{-}{-}{-}{-}{-}{-}{-}{-}{-}{-}{-}{-}{-}{-}{-}{-}{-}}

\FunctionTok{library}\NormalTok{(ggplot2)}
\FunctionTok{ggplot}\NormalTok{(base\_final\_mujeres, }\FunctionTok{aes}\NormalTok{(}\AttributeTok{x =}\NormalTok{ age, }\AttributeTok{y =}\NormalTok{ mxm, }\AttributeTok{group =} \FunctionTok{factor}\NormalTok{(year), }\AttributeTok{color =} \FunctionTok{factor}\NormalTok{(year))) }\SpecialCharTok{+}
  \FunctionTok{geom\_line}\NormalTok{() }\SpecialCharTok{+}
  \FunctionTok{labs}\NormalTok{(}
    \AttributeTok{title =} \StringTok{"Tasas de mortalidad en mujeres por edad y año"}\NormalTok{,}
    \AttributeTok{x =} \StringTok{"Edad"}\NormalTok{,}
    \AttributeTok{y =} \StringTok{"Tasa de mortalidad (mxM)"}\NormalTok{,}
    \AttributeTok{color =} \StringTok{"Año"}
\NormalTok{  ) }\SpecialCharTok{+}
  \FunctionTok{theme\_minimal}\NormalTok{()}
\end{Highlighting}
\end{Shaded}

\includegraphics{Proyecto-LC-completo_files/figure-latex/unnamed-chunk-13-1.pdf}

\begin{Shaded}
\begin{Highlighting}[]
\CommentTok{\# Segunda grafica}

\FunctionTok{ggplot}\NormalTok{(base\_final\_mujeres, }\FunctionTok{aes}\NormalTok{(}\AttributeTok{x =}\NormalTok{ age, }\AttributeTok{y =}\NormalTok{ mxm, }\AttributeTok{group =}\NormalTok{ year, }\AttributeTok{color =} \FunctionTok{factor}\NormalTok{(year))) }\SpecialCharTok{+}
  \FunctionTok{geom\_line}\NormalTok{() }\SpecialCharTok{+}
  \FunctionTok{scale\_y\_log10}\NormalTok{() }\SpecialCharTok{+}
  \FunctionTok{labs}\NormalTok{(}
    \AttributeTok{title =} \StringTok{"Tasas de mortalidad (escala logarítmica) mujeres"}\NormalTok{,}
    \AttributeTok{x =} \StringTok{"Edad"}\NormalTok{,}
    \AttributeTok{y =} \StringTok{"mxm (escala log)"}
\NormalTok{  ) }\SpecialCharTok{+}
  \FunctionTok{theme\_minimal}\NormalTok{()}
\end{Highlighting}
\end{Shaded}

\includegraphics{Proyecto-LC-completo_files/figure-latex/unnamed-chunk-13-2.pdf}

\begin{Shaded}
\begin{Highlighting}[]
\CommentTok{\# Tercera gráfica}
\FunctionTok{ggplot}\NormalTok{(base\_final\_mujeres, }\FunctionTok{aes}\NormalTok{(}\AttributeTok{x =}\NormalTok{ year, }
                               \AttributeTok{y =} \FunctionTok{log}\NormalTok{(mxm),      }
                               \AttributeTok{group =}\NormalTok{ age, }
                               \AttributeTok{color =}\NormalTok{ age)) }\SpecialCharTok{+}
  \FunctionTok{geom\_line}\NormalTok{() }\SpecialCharTok{+}
  
  \FunctionTok{scale\_color\_viridis\_c}\NormalTok{(}\AttributeTok{option =} \StringTok{"plasma"}\NormalTok{) }\SpecialCharTok{+}  
  \FunctionTok{labs}\NormalTok{(}
    \AttributeTok{title =} \StringTok{"Tasas de mortalidad de mujeres (log) por año y edad"}\NormalTok{,}
    \AttributeTok{x =} \StringTok{"Año"}\NormalTok{,}
    \AttributeTok{y =} \StringTok{"Log de la tasa de mortalidad"}\NormalTok{,}
    \AttributeTok{color =} \StringTok{"Edad"}
\NormalTok{  ) }\SpecialCharTok{+}
  \FunctionTok{theme\_minimal}\NormalTok{()}
\end{Highlighting}
\end{Shaded}

\includegraphics{Proyecto-LC-completo_files/figure-latex/unnamed-chunk-13-3.pdf}

\begin{Shaded}
\begin{Highlighting}[]
\CommentTok{\# TASAS DE MORTALIDAD HOMBRES {-}{-}{-}{-}{-}{-}{-}{-}{-}{-}{-}{-}{-}{-}{-}{-}{-}{-}{-}{-}{-}{-}{-}{-}{-}{-}{-}{-}{-}{-}{-}{-}{-}{-}{-}{-}{-}{-}{-}{-}{-}{-}{-}{-}{-}{-}{-}{-}{-}{-}{-}{-}{-}}

\FunctionTok{library}\NormalTok{(ggplot2)}
\FunctionTok{ggplot}\NormalTok{(base\_final\_hombres, }\FunctionTok{aes}\NormalTok{(}\AttributeTok{x =}\NormalTok{ age, }\AttributeTok{y =}\NormalTok{ mx, }\AttributeTok{group =} \FunctionTok{factor}\NormalTok{(year), }\AttributeTok{color =} \FunctionTok{factor}\NormalTok{(year))) }\SpecialCharTok{+}
  \FunctionTok{geom\_line}\NormalTok{() }\SpecialCharTok{+}
  \FunctionTok{labs}\NormalTok{(}
    \AttributeTok{title =} \StringTok{"Tasas de mortalidad de hombres por edad y año"}\NormalTok{,}
    \AttributeTok{x =} \StringTok{"Edad"}\NormalTok{,}
    \AttributeTok{y =} \StringTok{"Tasa de mortalidad (mx)"}\NormalTok{,}
    \AttributeTok{color =} \StringTok{"Año"}
\NormalTok{  ) }\SpecialCharTok{+}
  \FunctionTok{theme\_minimal}\NormalTok{()}
\end{Highlighting}
\end{Shaded}

\includegraphics{Proyecto-LC-completo_files/figure-latex/unnamed-chunk-13-4.pdf}

\begin{Shaded}
\begin{Highlighting}[]
\CommentTok{\# Segunda grafica}

\FunctionTok{ggplot}\NormalTok{(base\_final\_hombres, }\FunctionTok{aes}\NormalTok{(}\AttributeTok{x =}\NormalTok{ age, }\AttributeTok{y =}\NormalTok{ mx, }\AttributeTok{group =}\NormalTok{ year, }\AttributeTok{color =} \FunctionTok{factor}\NormalTok{(year))) }\SpecialCharTok{+}
  \FunctionTok{geom\_line}\NormalTok{() }\SpecialCharTok{+}
  \FunctionTok{scale\_y\_log10}\NormalTok{() }\SpecialCharTok{+}
  \FunctionTok{labs}\NormalTok{(}
    \AttributeTok{title =} \StringTok{"Tasas de mortalidad hombres (escala logarítmica)"}\NormalTok{,}
    \AttributeTok{x =} \StringTok{"Edad"}\NormalTok{,}
    \AttributeTok{y =} \StringTok{"mx (escala log)"}
\NormalTok{  ) }\SpecialCharTok{+}
  \FunctionTok{theme\_minimal}\NormalTok{()}
\end{Highlighting}
\end{Shaded}

\includegraphics{Proyecto-LC-completo_files/figure-latex/unnamed-chunk-13-5.pdf}

\begin{Shaded}
\begin{Highlighting}[]
\CommentTok{\# Tercera gráfica}
\FunctionTok{ggplot}\NormalTok{(base\_final\_hombres, }\FunctionTok{aes}\NormalTok{(}\AttributeTok{x =}\NormalTok{ year, }
                               \AttributeTok{y =} \FunctionTok{log}\NormalTok{(mx),       }
                               \AttributeTok{group =}\NormalTok{ age, }
                               \AttributeTok{color =}\NormalTok{ age)) }\SpecialCharTok{+}
  \FunctionTok{geom\_line}\NormalTok{() }\SpecialCharTok{+}
 
  \FunctionTok{scale\_color\_viridis\_c}\NormalTok{(}\AttributeTok{option =} \StringTok{"plasma"}\NormalTok{) }\SpecialCharTok{+}  
  \FunctionTok{labs}\NormalTok{(}
    \AttributeTok{title =} \StringTok{"Tasas de mortalidad de hombres (log) por año y edad"}\NormalTok{,}
    \AttributeTok{x =} \StringTok{"Año"}\NormalTok{,}
    \AttributeTok{y =} \StringTok{"Log de la tasa de mortalidad"}\NormalTok{,}
    \AttributeTok{color =} \StringTok{"Edad"}
\NormalTok{  ) }\SpecialCharTok{+}
  \FunctionTok{theme\_minimal}\NormalTok{()}
\end{Highlighting}
\end{Shaded}

\includegraphics{Proyecto-LC-completo_files/figure-latex/unnamed-chunk-13-6.pdf}

\hypertarget{metodologuxeda}{%
\section{5. Metodología}\label{metodologuxeda}}

A continuación, se describe la metodología empleada para el análisis de
datos y la construcción del modelo de mortalidad:

\begin{enumerate}
\def\labelenumi{\arabic{enumi}.}
\tightlist
\item
  \textbf{Recopilación y fuente de datos}

  \begin{itemize}
  \tightlist
  \item
    Se reunieron registros históricos de mortalidad, específicamente de
    \href{https://www.ssc.wisc.edu/cdha/latinmortality2/?page_id=28}{Latin
    Mortality Database (n.d.)} y
    \href{https://uisp.insp.mx/wp/index.php/tablas-de-vida/}{Tablas de
    vida (UISP, n.d.)}.\\
  \item
    Estos conjuntos de datos abarcan distintos periodos, desde finales
    del siglo XIX hasta inicios del siglo XXI, y contienen información
    relativa a tasas de mortalidad, número de sobrevivientes y años
    vividos, tanto para hombres como para mujeres.
  \end{itemize}
\item
  \textbf{Limpieza y preparación de la información}

  \begin{itemize}
  \tightlist
  \item
    Se revisó la estructura de los datos con funciones como
    \texttt{str()}, \texttt{head()} y \texttt{summary()} para
    identificar inconsistencias o errores de formato.\\
  \item
    Se corrigieron valores no válidos y se eliminaron o reemplazaron
    caracteres extra (por ejemplo, comas en campos numéricos). Además,
    se convirtieron las variables a tipos apropiados (numéricas,
    factores, etc.).\\
  \item
    Se unificaron las bases de datos provenientes de distintos periodos
    (por ejemplo, 1895-2005 y 2005-2013) mediante un proceso de
    \texttt{merge}, asegurando la correspondencia de las variables clave
    (edades, años y tasas de mortalidad).
  \end{itemize}
\item
  \textbf{Selección de variables relevantes}

  \begin{itemize}
  \tightlist
  \item
    Se identificaron las variables de mayor interés para el análisis
    actuarial, como \(m_x\) (tasa de mortalidad), \(l_x\)
    (sobrevivientes a la edad \(x\)), \(L_x\) (años vividos entre edades
    \(x\) y \(x+1\)) y \(T_x\) (suma acumulada de años vividos).\\
  \item
    Esta depuración de variables facilitó la construcción de las
    matrices requeridas para el modelo.
  \end{itemize}
\item
  \textbf{Transformación de datos en formato demográfico}

  \begin{itemize}
  \tightlist
  \item
    Con ayuda de funciones del paquete \textbf{\texttt{reshape2}} (por
    ejemplo, \texttt{acast}), se convirtieron los data frames en
    matrices, donde las filas corresponden a edades y las columnas a
    años.\\
  \item
    Estas matrices (de tasas, sobrevivientes y años vividos) se
    transformaron en objetos de la clase \textbf{\texttt{demogdata}}
    mediante la función \texttt{demogdata} del paquete
    \textbf{\texttt{demography}}, especificando los vectores de edades,
    años y el tipo de dato (mortality).
  \end{itemize}
\item
  \textbf{Construcción del modelo (Lee-Carter)}

  \begin{itemize}
  \tightlist
  \item
    Se aplicó el modelo \textbf{Lee-Carter} utilizando la función
    \texttt{lca()} del paquete \textbf{\texttt{demography}}.\\
  \item
    El modelo se ajustó en dos configuraciones diferentes, aprovechando
    tanto la matriz de \(L_x\) como la de \(l_x\) para comparar y
    contrastar los resultados.\\
  \item
    Se emplearon los comandos \texttt{summary()} y \texttt{plot()} para
    evaluar la calidad del ajuste y visualizar las tendencias de la
    mortalidad a lo largo del tiempo.
  \end{itemize}
\item
  \textbf{Software y herramientas empleadas}

  \begin{itemize}
  \tightlist
  \item
    \textbf{Paquetes en R:}

    \begin{itemize}
    \tightlist
    \item
      \textbf{\texttt{demography}} para la creación de objetos
      demográficos y aplicación del modelo Lee-Carter.\\
    \item
      \textbf{\texttt{reshape2}} (o \texttt{tidyr}) para la manipulación
      y reestructuración de datos.\\
    \item
      \textbf{\texttt{dplyr}, \texttt{ggplot2}, \texttt{readr}} y otros
      paquetes auxiliares para la limpieza, transformación y
      visualización de la información.
    \end{itemize}
  \end{itemize}
\end{enumerate}

Esta metodología aseguró la coherencia de los datos y la aplicación
adecuada del modelo Lee-Carter para el estudio de la mortalidad a lo
largo de distintos periodos históricos. A continuacion se explicará como
se utilizó cada línea de código para el caso de la base de datos de
hombres.

\begin{Shaded}
\begin{Highlighting}[]
\NormalTok{mx\_matrix }\OtherTok{\textless{}{-}} \FunctionTok{acast}\NormalTok{(base\_final\_hombres, age }\SpecialCharTok{\textasciitilde{}}\NormalTok{ year, }\AttributeTok{value.var =} \StringTok{"mx"}\NormalTok{)}

\NormalTok{mx\_matrix }\OtherTok{\textless{}{-}} \FunctionTok{as.matrix}\NormalTok{(mx\_matrix)  }
\NormalTok{mx\_matrix }\OtherTok{\textless{}{-}} \FunctionTok{apply}\NormalTok{(mx\_matrix, }\DecValTok{2}\NormalTok{, as.numeric)}
\NormalTok{base\_final\_hombres}\SpecialCharTok{$}\NormalTok{mx }\OtherTok{\textless{}{-}} \FunctionTok{as.numeric}\NormalTok{(base\_final\_hombres}\SpecialCharTok{$}\NormalTok{mx)}
\end{Highlighting}
\end{Shaded}

Creamos una matriz correspondiente a la tasa de mortalidad. En esta
matriz, las filas representan las edades y las columnas los años. Cada
celda contiene el valor de la tasa de mortalidad (mx) correspondiente a
esa combinación

\begin{Shaded}
\begin{Highlighting}[]
\NormalTok{Lx\_matrix }\OtherTok{\textless{}{-}} \FunctionTok{acast}\NormalTok{(base\_final\_hombres, age }\SpecialCharTok{\textasciitilde{}}\NormalTok{ year, }\AttributeTok{value.var =} \StringTok{"Lx"}\NormalTok{)}
\NormalTok{Lx\_matrix }\OtherTok{\textless{}{-}} \FunctionTok{as.matrix}\NormalTok{(Lx\_matrix)}
\NormalTok{Lx\_matrix }\OtherTok{\textless{}{-}} \FunctionTok{apply}\NormalTok{(Lx\_matrix, }\DecValTok{2}\NormalTok{, as.numeric)}
\FunctionTok{str}\NormalTok{(Lx\_matrix)}
\end{Highlighting}
\end{Shaded}

\begin{verbatim}
##  num [1:19, 1:34] 82268 253645 280694 267329 256098 ...
##  - attr(*, "dimnames")=List of 2
##   ..$ : NULL
##   ..$ : chr [1:34] "1895" "1897" "1900" "1905" ...
\end{verbatim}

Ahora transformamos la información correspondiente a \(L_x\) en una
matriz organizada por edad y año, y de asegurar que todos los datos sean
numéricos. Para comprobar la transformación, se muestra la estructura de
la matriz resultante.

\begin{Shaded}
\begin{Highlighting}[]
\NormalTok{lx\_matrix }\OtherTok{\textless{}{-}} \FunctionTok{acast}\NormalTok{(base\_final\_hombres, age }\SpecialCharTok{\textasciitilde{}}\NormalTok{ year, }\AttributeTok{value.var =} \StringTok{"lx"}\NormalTok{)}
\NormalTok{lx\_matrix }\OtherTok{\textless{}{-}} \FunctionTok{as.matrix}\NormalTok{(lx\_matrix)}
\NormalTok{lx\_matrix }\OtherTok{\textless{}{-}} \FunctionTok{apply}\NormalTok{(lx\_matrix, }\DecValTok{2}\NormalTok{, as.numeric)}
\FunctionTok{str}\NormalTok{(lx\_matrix)}
\end{Highlighting}
\end{Shaded}

\begin{verbatim}
##  num [1:19, 1:34] 100000 73535 58242 54417 52435 ...
##  - attr(*, "dimnames")=List of 2
##   ..$ : NULL
##   ..$ : chr [1:34] "1895" "1897" "1900" "1905" ...
\end{verbatim}

Por ultimo, convertimos los datos de \(l_x\) en una matriz, estructurada
de manera que cada fila representa una edad y cada columna un año. Se
aplica la conversión a formato numérico, y luego se utiliza
\texttt{str()} para revisar la estructura resultante de la matriz.

\begin{Shaded}
\begin{Highlighting}[]
\NormalTok{my\_data }\OtherTok{\textless{}{-}} \FunctionTok{demogdata}\NormalTok{(}
  \AttributeTok{data =}\NormalTok{ mx\_matrix,           }
  \AttributeTok{pop =}\NormalTok{ Lx\_matrix,}
  \AttributeTok{ages =} \FunctionTok{c}\NormalTok{(}\DecValTok{0}\NormalTok{, }\DecValTok{1}\NormalTok{, }\DecValTok{5}\NormalTok{, }\DecValTok{10}\NormalTok{, }\DecValTok{15}\NormalTok{, }\DecValTok{20}\NormalTok{, }\DecValTok{25}\NormalTok{, }\DecValTok{30}\NormalTok{, }\DecValTok{35}\NormalTok{, }\DecValTok{40}\NormalTok{, }\DecValTok{45}\NormalTok{, }\DecValTok{50}\NormalTok{, }\DecValTok{55}\NormalTok{, }\DecValTok{60}\NormalTok{, }\DecValTok{65}\NormalTok{, }\DecValTok{70}\NormalTok{, }\DecValTok{75}\NormalTok{, }\DecValTok{80}\NormalTok{, }\DecValTok{85}\NormalTok{),        }
  \AttributeTok{years =} \FunctionTok{c}\NormalTok{(}\DecValTok{1895}\NormalTok{,}\DecValTok{1897}\NormalTok{,}\DecValTok{1900}\NormalTok{,}\DecValTok{1905}\NormalTok{,}\DecValTok{1910}\NormalTok{,}\DecValTok{1915}\NormalTok{,}\DecValTok{1921}\NormalTok{,}\DecValTok{1925}\NormalTok{,}\DecValTok{1930}\NormalTok{,}\DecValTok{1935}\NormalTok{,}\DecValTok{1940}\NormalTok{,}\DecValTok{1945}\NormalTok{,}\DecValTok{1955}\NormalTok{,}\DecValTok{1965}\NormalTok{,}\DecValTok{1975}\NormalTok{,}\DecValTok{1985}\NormalTok{,}\DecValTok{1995}\NormalTok{,}\DecValTok{2005}\NormalTok{,}\DecValTok{2006}\NormalTok{,}\DecValTok{2007}\NormalTok{,}\DecValTok{2008}\NormalTok{,}\DecValTok{2009}\NormalTok{,}\DecValTok{2010}\NormalTok{,}\DecValTok{2011}\NormalTok{,}\DecValTok{2012}\NormalTok{,}\DecValTok{2013}\NormalTok{,}\DecValTok{2014}\NormalTok{,}\DecValTok{2015}\NormalTok{,}\DecValTok{2016}\NormalTok{,}\DecValTok{2017}\NormalTok{,}\DecValTok{2018}\NormalTok{,}\DecValTok{2019}\NormalTok{,}\DecValTok{2020}\NormalTok{,}\DecValTok{2021}\NormalTok{),   }
  \AttributeTok{type =} \StringTok{"mortality"}\NormalTok{,  }
  \AttributeTok{label =} \StringTok{"Mis datos"}\NormalTok{,}
  \AttributeTok{name =} \StringTok{"Hombres"}
\NormalTok{)}
\end{Highlighting}
\end{Shaded}

Las anteriores líneas de código crean un objeto demográfico llamado
\texttt{my\_data} utilizando la función \texttt{demogdata}. Básicamente,
se establecen los elementos esenciales para el análisis, incluyendo la
matriz de tasas de mortalidad, la población (o medida relacionada), y
los vectores de edades y años. Además, se especifica el tipo de dato (en
este caso, ``mortality'') y se asignan etiquetas para identificar el
conjunto de datos. Este objeto se utiliza posteriormente para realizar
análisis demográficos basados en \(L_x\).

\begin{Shaded}
\begin{Highlighting}[]
\CommentTok{\# Con matriz lx}
\NormalTok{my\_data2 }\OtherTok{\textless{}{-}} \FunctionTok{demogdata}\NormalTok{(}
  \AttributeTok{data =}\NormalTok{ mx\_matrix,           }
  \AttributeTok{pop =}\NormalTok{ lx\_matrix,}
  \AttributeTok{ages =} \FunctionTok{c}\NormalTok{(}\DecValTok{0}\NormalTok{, }\DecValTok{1}\NormalTok{, }\DecValTok{5}\NormalTok{, }\DecValTok{10}\NormalTok{, }\DecValTok{15}\NormalTok{, }\DecValTok{20}\NormalTok{, }\DecValTok{25}\NormalTok{, }\DecValTok{30}\NormalTok{, }\DecValTok{35}\NormalTok{, }\DecValTok{40}\NormalTok{, }\DecValTok{45}\NormalTok{, }\DecValTok{50}\NormalTok{, }\DecValTok{55}\NormalTok{, }\DecValTok{60}\NormalTok{, }\DecValTok{65}\NormalTok{, }\DecValTok{70}\NormalTok{, }\DecValTok{75}\NormalTok{, }\DecValTok{80}\NormalTok{, }\DecValTok{85}\NormalTok{),       }
  \AttributeTok{years =} \FunctionTok{c}\NormalTok{(}\DecValTok{1895}\NormalTok{,}\DecValTok{1897}\NormalTok{,}\DecValTok{1900}\NormalTok{,}\DecValTok{1905}\NormalTok{,}\DecValTok{1910}\NormalTok{,}\DecValTok{1915}\NormalTok{,}\DecValTok{1921}\NormalTok{,}\DecValTok{1925}\NormalTok{,}\DecValTok{1930}\NormalTok{,}\DecValTok{1935}\NormalTok{,}\DecValTok{1940}\NormalTok{,}\DecValTok{1945}\NormalTok{,}\DecValTok{1955}\NormalTok{,}\DecValTok{1965}\NormalTok{,}\DecValTok{1975}\NormalTok{,}\DecValTok{1985}\NormalTok{,}\DecValTok{1995}\NormalTok{,}\DecValTok{2005}\NormalTok{,}\DecValTok{2006}\NormalTok{,}\DecValTok{2007}\NormalTok{,}\DecValTok{2008}\NormalTok{,}\DecValTok{2009}\NormalTok{,}\DecValTok{2010}\NormalTok{,}\DecValTok{2011}\NormalTok{,}\DecValTok{2012}\NormalTok{,}\DecValTok{2013}\NormalTok{,}\DecValTok{2014}\NormalTok{,}\DecValTok{2015}\NormalTok{,}\DecValTok{2016}\NormalTok{,}\DecValTok{2017}\NormalTok{,}\DecValTok{2018}\NormalTok{,}\DecValTok{2019}\NormalTok{,}\DecValTok{2020}\NormalTok{,}\DecValTok{2021}\NormalTok{),   }
  \AttributeTok{type =} \StringTok{"mortality"}\NormalTok{,  }
  \AttributeTok{label =} \StringTok{"Mis datos"}\NormalTok{,}
  \AttributeTok{name =} \StringTok{"Hombres"}
\NormalTok{)}
\end{Highlighting}
\end{Shaded}

El siguiente bloque de código crea un objeto demográfico llamado
\texttt{my\_data2} utilizando la función \texttt{demogdata}. En este
caso, se utiliza la matriz de tasas de mortalidad (\texttt{mx\_matrix})
y la matriz \texttt{lx\_matrix} para la población. Se establecen los
vectores de edades y años, se define el tipo de dato (``mortality'') y
se asignan etiquetas al conjunto de datos. Este objeto servirá para
llevar a cabo análisis demográficos utilizando la información basada en
la variable \(l_x\).

\begin{Shaded}
\begin{Highlighting}[]
\CommentTok{\# Realizamos el ajuste}
\NormalTok{fit\_lc }\OtherTok{\textless{}{-}} \FunctionTok{lca}\NormalTok{(my\_data, }\AttributeTok{series =} \StringTok{"Hombres"}\NormalTok{, }\AttributeTok{adjust =} \StringTok{"e0"}\NormalTok{)}
\FunctionTok{summary}\NormalTok{(fit\_lc)}
\end{Highlighting}
\end{Shaded}

\begin{verbatim}
## Lee-Carter analysis
## 
## Call: lca(data = my_data, series = "Hombres", adjust = "e0") 
## 
## Adjustment method: e0
## Region: Mis datos
## Years in fit: 1895 - 2021
## Ages in fit: 0 - 85 
## 
## Percentage variation explained: 97.8%
## 
## ERROR MEASURES BASED ON MORTALITY RATES
## 
## Averages across ages:
##       ME      MSE      MPE     MAPE 
## -0.00056  0.00011  0.03198  0.12531 
## 
## Averages across years:
##       IE      ISE      IPE     IAPE 
## -0.03518  0.00696  2.21575  9.81128 
## 
## 
## ERROR MEASURES BASED ON LOG MORTALITY RATES
## 
## Averages across ages:
##      ME     MSE     MPE    MAPE 
## 0.00843 0.03643 0.00141 0.02939 
## 
## Averages across years:
##      IE     ISE     IPE    IAPE 
## 0.66557 2.60205 0.07310 2.29206
\end{verbatim}

\begin{Shaded}
\begin{Highlighting}[]
\FunctionTok{plot}\NormalTok{(fit\_lc)}
\end{Highlighting}
\end{Shaded}

\includegraphics{Proyecto-LC-completo_files/figure-latex/unnamed-chunk-19-1.pdf}

En este bloque de código se aplica el modelo Lee-Carter para analizar
las tasas de mortalidad en la serie de datos de hombres. Se realizan dos
ajustes, cada uno utilizando un conjunto de datos distinto:

\begin{itemize}
\item
  \textbf{Con \texttt{my\_data}:}\\
  Se ajusta el modelo Lee-Carter a los datos contenidos en
  \texttt{my\_data} y se obtienen tanto un resumen como una
  representación gráfica de los resultados.
\item
  \textbf{Con \texttt{my\_data2}:}\\
  Se repite el proceso utilizando \texttt{my\_data2}, lo que permite
  comparar cómo se comporta el modelo cuando se utiliza otra medida de
  población.
\end{itemize}

Esta comparación nos ayuda a evaluar el rendimiento y la consistencia
del modelo bajo diferentes variables dentro de los datos.

El análisis para la base de datos de mujeres se realiza de forma análoga
al que aplicamos para la base de datos de hombres. Dado que los
procedimientos y métodos empleados son equivalentes, a continuación se
presentan únicamente las líneas de código utilizadas para procesar y
analizar los datos femeninos.

\begin{Shaded}
\begin{Highlighting}[]
\CommentTok{\# Corremos el modelo}
\CommentTok{\# Generamos las matrices}

\NormalTok{mxm\_matrix }\OtherTok{\textless{}{-}} \FunctionTok{acast}\NormalTok{(base\_final\_mujeres, age }\SpecialCharTok{\textasciitilde{}}\NormalTok{ year, }\AttributeTok{value.var =} \StringTok{"mxm"}\NormalTok{)}


\NormalTok{mxm\_matrix }\OtherTok{\textless{}{-}} \FunctionTok{as.matrix}\NormalTok{(mxm\_matrix)  }
\NormalTok{mxm\_matrix }\OtherTok{\textless{}{-}} \FunctionTok{apply}\NormalTok{(mxm\_matrix, }\DecValTok{2}\NormalTok{, as.numeric)}
\NormalTok{base\_final\_mujeres}\SpecialCharTok{$}\NormalTok{mxm }\OtherTok{\textless{}{-}} \FunctionTok{as.numeric}\NormalTok{(base\_final\_mujeres}\SpecialCharTok{$}\NormalTok{mxm)}

\CommentTok{\# \textquotesingle{}apply\textquotesingle{} recorre cada columna y la convierte a numeric, devolviendo una matriz}

\FunctionTok{str}\NormalTok{(mxm\_matrix)}
\end{Highlighting}
\end{Shaded}

\begin{verbatim}
##  num [1:19, 1:34] 0.27222 0.22613 0.01243 0.00619 0.00925 ...
##  - attr(*, "dimnames")=List of 2
##   ..$ : NULL
##   ..$ : chr [1:34] "1895" "1897" "1900" "1905" ...
\end{verbatim}

\begin{Shaded}
\begin{Highlighting}[]
\CommentTok{\# Corremos las otras matrices}
\NormalTok{Lxm\_matrix }\OtherTok{\textless{}{-}} \FunctionTok{acast}\NormalTok{(base\_final\_mujeres, age }\SpecialCharTok{\textasciitilde{}}\NormalTok{ year, }\AttributeTok{value.var =} \StringTok{"Lxm"}\NormalTok{)}
\NormalTok{Lxm\_matrix }\OtherTok{\textless{}{-}} \FunctionTok{as.matrix}\NormalTok{(Lxm\_matrix)}
\NormalTok{Lxm\_matrix }\OtherTok{\textless{}{-}} \FunctionTok{apply}\NormalTok{(Lxm\_matrix, }\DecValTok{2}\NormalTok{, as.numeric)}
\FunctionTok{str}\NormalTok{(Lxm\_matrix)}
\end{Highlighting}
\end{Shaded}

\begin{verbatim}
##  num [1:19, 1:34] 84510 263995 293708 281295 270752 ...
##  - attr(*, "dimnames")=List of 2
##   ..$ : NULL
##   ..$ : chr [1:34] "1895" "1897" "1900" "1905" ...
\end{verbatim}

\begin{Shaded}
\begin{Highlighting}[]
\NormalTok{lxm\_matrix }\OtherTok{\textless{}{-}} \FunctionTok{acast}\NormalTok{(base\_final\_mujeres, age }\SpecialCharTok{\textasciitilde{}}\NormalTok{ year, }\AttributeTok{value.var =} \StringTok{"lxm"}\NormalTok{)}
\NormalTok{lxm\_matrix }\OtherTok{\textless{}{-}} \FunctionTok{as.matrix}\NormalTok{(lxm\_matrix)}
\NormalTok{lxm\_matrix }\OtherTok{\textless{}{-}} \FunctionTok{apply}\NormalTok{(lxm\_matrix, }\DecValTok{2}\NormalTok{, as.numeric)}
\FunctionTok{str}\NormalTok{(lxm\_matrix)}
\end{Highlighting}
\end{Shaded}

\begin{verbatim}
##  num [1:19, 1:34] 100000 76169 60754 57096 55353 ...
##  - attr(*, "dimnames")=List of 2
##   ..$ : NULL
##   ..$ : chr [1:34] "1895" "1897" "1900" "1905" ...
\end{verbatim}

\begin{Shaded}
\begin{Highlighting}[]
\CommentTok{\# Ajustamos el modelo }

\NormalTok{my\_data\_mujeres }\OtherTok{\textless{}{-}} \FunctionTok{demogdata}\NormalTok{(}
  \AttributeTok{data =}\NormalTok{ mxm\_matrix,           }
  \AttributeTok{pop =}\NormalTok{ Lxm\_matrix,}
  \AttributeTok{ages =} \FunctionTok{c}\NormalTok{(}\DecValTok{0}\NormalTok{, }\DecValTok{1}\NormalTok{, }\DecValTok{5}\NormalTok{, }\DecValTok{10}\NormalTok{, }\DecValTok{15}\NormalTok{, }\DecValTok{20}\NormalTok{, }\DecValTok{25}\NormalTok{, }\DecValTok{30}\NormalTok{, }\DecValTok{35}\NormalTok{, }\DecValTok{40}\NormalTok{, }\DecValTok{45}\NormalTok{, }\DecValTok{50}\NormalTok{, }\DecValTok{55}\NormalTok{, }\DecValTok{60}\NormalTok{, }\DecValTok{65}\NormalTok{, }\DecValTok{70}\NormalTok{, }\DecValTok{75}\NormalTok{, }\DecValTok{80}\NormalTok{, }\DecValTok{85}\NormalTok{),       }
  \AttributeTok{years =} \FunctionTok{c}\NormalTok{(}\DecValTok{1895}\NormalTok{,}\DecValTok{1897}\NormalTok{,}\DecValTok{1900}\NormalTok{,}\DecValTok{1905}\NormalTok{,}\DecValTok{1910}\NormalTok{,}\DecValTok{1915}\NormalTok{,}\DecValTok{1921}\NormalTok{,}\DecValTok{1925}\NormalTok{,}\DecValTok{1930}\NormalTok{,}\DecValTok{1935}\NormalTok{,}\DecValTok{1940}\NormalTok{,}\DecValTok{1945}\NormalTok{,}\DecValTok{1955}\NormalTok{,}\DecValTok{1965}\NormalTok{,}\DecValTok{1975}\NormalTok{,}\DecValTok{1985}\NormalTok{,}\DecValTok{1995}\NormalTok{,}\DecValTok{2005}\NormalTok{,}\DecValTok{2006}\NormalTok{,}\DecValTok{2007}\NormalTok{,}\DecValTok{2008}\NormalTok{,}\DecValTok{2009}\NormalTok{,}\DecValTok{2010}\NormalTok{,}\DecValTok{2011}\NormalTok{,}\DecValTok{2012}\NormalTok{,}\DecValTok{2013}\NormalTok{,}\DecValTok{2014}\NormalTok{,}\DecValTok{2015}\NormalTok{,}\DecValTok{2016}\NormalTok{,}\DecValTok{2017}\NormalTok{,}\DecValTok{2018}\NormalTok{,}\DecValTok{2019}\NormalTok{,}\DecValTok{2020}\NormalTok{,}\DecValTok{2021}\NormalTok{),   }
  \AttributeTok{type =} \StringTok{"mortality"}\NormalTok{,  }
  \AttributeTok{label =} \StringTok{"Mis datos"}\NormalTok{,}
  \AttributeTok{name =} \StringTok{"Mujeres"}
\NormalTok{)}
\NormalTok{my\_data\_mujeres2 }\OtherTok{\textless{}{-}} \FunctionTok{demogdata}\NormalTok{(}
  \AttributeTok{data =}\NormalTok{ mxm\_matrix,          }
  \AttributeTok{pop =}\NormalTok{ lxm\_matrix,}
  \AttributeTok{ages =} \FunctionTok{c}\NormalTok{(}\DecValTok{0}\NormalTok{, }\DecValTok{1}\NormalTok{, }\DecValTok{5}\NormalTok{, }\DecValTok{10}\NormalTok{, }\DecValTok{15}\NormalTok{, }\DecValTok{20}\NormalTok{, }\DecValTok{25}\NormalTok{, }\DecValTok{30}\NormalTok{, }\DecValTok{35}\NormalTok{, }\DecValTok{40}\NormalTok{, }\DecValTok{45}\NormalTok{, }\DecValTok{50}\NormalTok{, }\DecValTok{55}\NormalTok{, }\DecValTok{60}\NormalTok{, }\DecValTok{65}\NormalTok{, }\DecValTok{70}\NormalTok{, }\DecValTok{75}\NormalTok{, }\DecValTok{80}\NormalTok{, }\DecValTok{85}\NormalTok{),       }
  \AttributeTok{years =} \FunctionTok{c}\NormalTok{(}\DecValTok{1895}\NormalTok{,}\DecValTok{1897}\NormalTok{,}\DecValTok{1900}\NormalTok{,}\DecValTok{1905}\NormalTok{,}\DecValTok{1910}\NormalTok{,}\DecValTok{1915}\NormalTok{,}\DecValTok{1921}\NormalTok{,}\DecValTok{1925}\NormalTok{,}\DecValTok{1930}\NormalTok{,}\DecValTok{1935}\NormalTok{,}\DecValTok{1940}\NormalTok{,}\DecValTok{1945}\NormalTok{,}\DecValTok{1955}\NormalTok{,}\DecValTok{1965}\NormalTok{,}\DecValTok{1975}\NormalTok{,}\DecValTok{1985}\NormalTok{,}\DecValTok{1995}\NormalTok{,}\DecValTok{2005}\NormalTok{,}\DecValTok{2006}\NormalTok{,}\DecValTok{2007}\NormalTok{,}\DecValTok{2008}\NormalTok{,}\DecValTok{2009}\NormalTok{,}\DecValTok{2010}\NormalTok{,}\DecValTok{2011}\NormalTok{,}\DecValTok{2012}\NormalTok{,}\DecValTok{2013}\NormalTok{,}\DecValTok{2014}\NormalTok{,}\DecValTok{2015}\NormalTok{,}\DecValTok{2016}\NormalTok{,}\DecValTok{2017}\NormalTok{,}\DecValTok{2018}\NormalTok{,}\DecValTok{2019}\NormalTok{,}\DecValTok{2020}\NormalTok{,}\DecValTok{2021}\NormalTok{),   }
  \AttributeTok{type =} \StringTok{"mortality"}\NormalTok{, }
  \AttributeTok{label =} \StringTok{"Mis datos"}\NormalTok{,}
  \AttributeTok{name =} \StringTok{"Mujeres"}
\NormalTok{)}

\CommentTok{\# Realizamos el ajuste correspondiente}
\NormalTok{fit\_lc\_mujeres }\OtherTok{\textless{}{-}} \FunctionTok{lca}\NormalTok{(my\_data\_mujeres, }\AttributeTok{series =} \StringTok{"Mujeres"}\NormalTok{, }\AttributeTok{adjust =} \StringTok{"e0"}\NormalTok{)}
\FunctionTok{summary}\NormalTok{(fit\_lc\_mujeres)}
\end{Highlighting}
\end{Shaded}

\begin{verbatim}
## Lee-Carter analysis
## 
## Call: lca(data = my_data_mujeres, series = "Mujeres", adjust = "e0") 
## 
## Adjustment method: e0
## Region: Mis datos
## Years in fit: 1895 - 2021
## Ages in fit: 0 - 85 
## 
## Percentage variation explained: 98.5%
## 
## ERROR MEASURES BASED ON MORTALITY RATES
## 
## Averages across ages:
##       ME      MSE      MPE     MAPE 
## -0.00054  0.00010  0.03130  0.11586 
## 
## Averages across years:
##       IE      ISE      IPE     IAPE 
## -0.03506  0.00633  2.37309  9.37368 
## 
## 
## ERROR MEASURES BASED ON LOG MORTALITY RATES
## 
## Averages across ages:
##      ME     MSE     MPE    MAPE 
## 0.01360 0.03026 0.00061 0.02705 
## 
## Averages across years:
##      IE     ISE     IPE    IAPE 
## 1.11345 2.30286 0.01271 2.12253
\end{verbatim}

\begin{Shaded}
\begin{Highlighting}[]
\FunctionTok{plot}\NormalTok{(fit\_lc\_mujeres)}
\end{Highlighting}
\end{Shaded}

\includegraphics{Proyecto-LC-completo_files/figure-latex/unnamed-chunk-20-1.pdf}

\clearpage

\hypertarget{comparaciuxf3n-de-componentes-del-modelo-lee-carter-hombres-vs-mujeres}{%
\subsection{5.1. Comparación de componentes del modelo Lee-Carter
Hombres vs
Mujeres}\label{comparaciuxf3n-de-componentes-del-modelo-lee-carter-hombres-vs-mujeres}}

\hypertarget{hombres}{%
\subsubsection{Hombres}\label{hombres}}

\begin{Shaded}
\begin{Highlighting}[]
\FunctionTok{plot}\NormalTok{(fit\_lc)}
\end{Highlighting}
\end{Shaded}

\includegraphics{Proyecto-LC-completo_files/figure-latex/unnamed-chunk-21-1.pdf}

\hypertarget{interpretaciones}{%
\subsubsection{Interpretaciones}\label{interpretaciones}}

\begin{itemize}
\item
  \(\alpha_x\): Efectos principales. La tasa de mortalidad aumenta
  significativamente con la edad. La mortalidad por edad es más
  pronunciada conforme los hombres envejecen. Podemos observar un
  aumento más pronunciado en la mortalidad para los hombres de edad
  avanzada, en comparación con las mujeres, lo cual nos indica que las
  tasas de mortalidad en los hombres son más altas en edades más
  avanzadas.
\item
  \(\beta_x\): Interacción. La curva para los hombres indica que la
  mortalidad en los jóvenes ha disminuido, mientras que para las edades
  más avanzadas, hay un descenso en las tasas de mortalidad, lo que nos
  indica que son menos pronunciadas en los últimos años.
\item
  \(\kappa_x\): Tendencia temporal ajustada. Nos muestra una caída
  importante en las tasas de mortalidad desde a principios del siglo XX,
  esto nos indica una mejora en el sistema de salud pública y los
  avances médicos que ha tenido México.
\end{itemize}

\clearpage

\hypertarget{mujeres}{%
\subsubsection{Mujeres}\label{mujeres}}

\begin{Shaded}
\begin{Highlighting}[]
\FunctionTok{plot}\NormalTok{(fit\_lc\_mujeres)}
\end{Highlighting}
\end{Shaded}

\includegraphics{Proyecto-LC-completo_files/figure-latex/unnamed-chunk-22-1.pdf}

\hypertarget{interpretaciones-1}{%
\subsubsection{Interpretaciones:}\label{interpretaciones-1}}

\begin{itemize}
\item
  \(\alpha_x\): Efectos principales. La tasa de mortalidad aumenta
  significativamente con la edad, pero de manera menos pronunciada en
  comparación con la de los hombres. Las mujeres tienden a tener una
  tasa de mortalidad más baja.
\item
  \(\beta_x\): Interacción. La curva de mortalidad para las mujeres en
  uno de los periodos se suaviza, lo que nos indica que existe una
  tendencia más estable a través del tiempo.
\item
  \(\kappa_x\): Tendencia temporal ajustada. La tasa de motalidad en el
  caso de las mujeres presenta una caída más moderada, esto nos sugiere
  que puede estar relacionada con factores sociales, acceso a la salud o
  incluso diferencias de género, entre otros factores.
\end{itemize}

\clearpage

\hypertarget{conclusiones}{%
\subsection{5.2. Conclusiones}\label{conclusiones}}

La diferencia en la esperanza de vida implica que las mujeres
necesitarán más recursos a lo largo de su vida, ya que presentan mayor
longevidad. Los sistemas de pensiones en México deberían de adaptarse a
este hecho.

Sabemos que México enfrenta un envejecimiento poblacional, lo que
conlleva a una mayor demanda de servicios de salud y pensiones. Es
gracias a este modelo que podemos crear medidas para el cuidado de
bienestar social.

\clearpage

\hypertarget{propuesta-de-modelo}{%
\section{6. Propuesta de Modelo}\label{propuesta-de-modelo}}

\hypertarget{descripciuxf3n-del-modelo-propuesto}{%
\subsection{6.1. Descripción del modelo
propuesto}\label{descripciuxf3n-del-modelo-propuesto}}

El modelo Lee-Carter es una metodología estadística que fue desarrollada
en 1992 por el demógrafo Raymond Lee y el economista Robert Carter, fue
diseñada con el propósito de modelar y predecir las tasas de mortalidad
de una población a lo largo del tiempo.

Este modelo se ha convertido en una herramienta esencial en la
demografía, especialmente en el análisis y proyección de la mortalidad
de una población a lo largo de los años.El objetivo principal es
capturar y proyectar las tasas de mortalidad de una población para
distintos grupos de años. Esto lo hace apoyándose de datos históricos de
mortalidad y de población.

Es útil para entender tendencias de mortalidad a largo plazo, esto lo
hace útil en áreas como pensiones, salúd pública y estudios de
longevidad.

\hypertarget{fuxf3rmula-del-modelo-lee-carter}{%
\subsubsection{6.2 Fórmula del Modelo
Lee-Carter}\label{fuxf3rmula-del-modelo-lee-carter}}

\[
\log(m_{x,t}) = \alpha_x + \beta_x \kappa_t + \epsilon_{x,t}
\]

Donde los parámetros son los siguientes:

\begin{itemize}
\tightlist
\item
  \(\alpha_x\): \textbf{Tendencia temporal} para el grupo de edad \(x\).
  Este parámetro captura la evolución general de la mortalidad dentro de
  la población a lo largo del tiempo.
\item
  \(\beta_x\): \textbf{Tasa de mortalidad base} para el grupo de edad
  \(x\) en el período base. Representa la mortalidad para cada grupo de
  edad suponiendo que no hubiera cambios en el tiempo.
\item
  \(\kappa_t\): \textbf{Componente de interacción entre edad y tiempo},
  describe cómo las tasas de mortalidad de cada grupo de edad cambian a
  través del tiempo.
\item
  \(\epsilon_{x,t}\): \textbf{Término de error} asociado a cada
  observación de las tasas de mortalidad, que captura las variaciones no
  explicadas por los otros parámetros que hay en el modelo.
\end{itemize}

\hypertarget{ventajas-del-modelo}{%
\subsection{6.3. Ventajas del modelo}\label{ventajas-del-modelo}}

\textbf{1.Simplicidad y eficacia:} además de que es un modelo simple, el
modelo Lee-Carter es eficaz para describir y proyectar las tasas de
mortalidad.

\textbf{2. Flexibilidad:} puede adaptarse a distintas poblaciones y
datos de mortalidad, también se puede adaptar a distintas edades y
períodos en el tiempo.

\textbf{3. Proyecciones a largo plazo:} permite realizar proyecciones de
mortalidad a largo plazo, lo que lo hace útil para planificar
estrategias de salúd pública.

\hypertarget{limitaciones-del-modelo}{%
\subsection{6.4. Limitaciones del
modelo}\label{limitaciones-del-modelo}}

\textbf{1. Linealidad del modelo:} el modelo asume que la mortalidad
sigue una tendencia logarítmica a través del tiempo. Esto puede no ser
completamente realista, ya que pueden haber eventos que alteren las
tasas de mortalidad.

\textbf{2. No captura las complejidades:} como es un modelo simple, no
logra capturar las complejidades más detalladas de las tasas de
mortalidad, tales como los impactos de eventos demográficos
significativos o factores sociales.

\hypertarget{justificaciuxf3n-de-la-elecciuxf3n-del-modelo}{%
\subsection{6.5. Justificación de la elección del
modelo}\label{justificaciuxf3n-de-la-elecciuxf3n-del-modelo}}

Elegimos el modelo Lee-Carter debido a que es ideal para proyectar las
tasas de mortalidad a largo plazo. Su capacidad para descomponer las
tasas de mortalidad en tres componentes clave la tasa base por edad, la
tendencia temporal y la interacción entre edad y tiempo, lo que lo
convierte en una herramienta poderosa para la proyección precisa de la
mortalidad en cualquier población. Además, su flexibilidad permite
modelar diferentes escenarios y adaptarlo a diversas condiciones
demográficas, lo que lo hace adecuado para México.

Decidimos aplicar este modelo a México debido a que proporciona una
herramienta útil para anticipar las tendencias futuras de mortalidad en
el país. Las proyecciones obtenidas permitirán planificar con mayor
precisión los recursos necesarios en áreas críticas como salud,
pensiones y bienestar social. En particular, el modelo tiene
aplicaciones clave dentro del sistema de pensiones y la seguridad
social, ya que facilita la proyección del número de personas que
alcanzarán la edad de jubilación y la estimación de las tasas de
mortalidad de ese grupo.

Además, creemos que con este trabajo estamos contribuyendo
significativamente al área demográfica en México, ya que, al investigar
el modelo Lee-Carter aplicado a datos históricos del país, hemos notado
que existen pocos estudios que aborden este tema desde esta perspectiva.
Con este análisis, esperamos aportar un valor adicional a la comprensión
de la mortalidad en México y ofrecer una herramienta útil para futuras
investigaciones y también en políticas públicas.

\hypertarget{supuestos-del-modelo-lee-carter}{%
\subsection{6.6. Supuestos del modelo
Lee-Carter}\label{supuestos-del-modelo-lee-carter}}

\textbf{1. Linealidad temporal:} las tasas de mortalidad siguen una
tendencia logarítmica a través del tiepo.

\textbf{2. Independencia de las tasas de mortalidad:} las tasas de
mortalidad por edad son independientes entre sí.

\textbf{3. Interacción entre edad y tiempo:} existe un término de
interacción que modela cómo las tasas de mortalidad de cada grupo de
edad van cambiando con el tiempo.

\textbf{4. Homogeneidad dentro de los grupos de edad:} se asume que
todos los individuos dentro de un grupo de edad tienen la misma tasa de
mortalidad.

\textbf{5. Ausencia de choques externos:} no se consideran eventos fuera
de lo normal (como pandemias o guerras) que alteren considerablemente
las tasas de mortalidad.

\textbf{6. Normalidad del error:} el término de error
(\(\epsilon_{x,t}\)) que captura las desviaciones en las tasas de
mortalidad sigue una distribución normal.

\textbf{7. Estabilidad de las proyecciones:} suponemos que las
tendencias históricas de mortalidad continúan de forma estable en el
futuro.

\clearpage

\hypertarget{proceso-de-estimaciuxf3n}{%
\subsection{6.7. Proceso de estimación}\label{proceso-de-estimaciuxf3n}}

La estimación del modelo se realiza a través de un análisis de
componentes principales (PCA) o también de descomposición en valores
singulares (SVD).

Lo primero que se hace es realizar el logaritmo de las tasas de
mortalidad (mx), ya que el modelo proyecta de esta manera la mortalidad.
Después se realiza la descomposición de los componentes para estimar
(alfa) , (beta) y (kappa).

La optimización de estos parámetros obtenidos se realiza a través de un
proceso de optimización, de manera que se minimizan la diferencia entre
las tasas de mortalidad observadas y las predicciones generadas por el
modelo.

\hypertarget{paruxe1metros-del-modelo-lee-carter}{%
\subsection{6.8. Parámetros del modelo
Lee-Carter}\label{paruxe1metros-del-modelo-lee-carter}}

\hypertarget{paruxe1metro-alfa}{%
\subsubsection{Parámetro alfa}\label{paruxe1metro-alfa}}

\begin{itemize}
\tightlist
\item
  \(\alpha_x\): \textbf{Tendencia temporal} para el grupo de edad \(x\).
  Este parámetro captura la evolución general de la mortalidad dentro de
  la población a lo largo del tiempo.
\end{itemize}

\hypertarget{estimaciuxf3n}{%
\subsubsection{Estimación}\label{estimaciuxf3n}}

\[
\alpha_t = \frac{1}{X} \sum_{x=1}^{X} \left(\log(m_{x,t}) - \beta_x \right)
\]

Muestra cómo las tasas de mortalidad para cada evolucionan de manera
distitna en cada año. Captura la interacción específica entre la edad x
y el año t.

Se puede calcular a partir de la descomposición de las tasas de
mortalidad observadas, utilizando un proceso de optimización para
reducir los errores en la predicción.

\hypertarget{paruxe1metro-beta}{%
\subsubsection{Parámetro beta}\label{paruxe1metro-beta}}

\begin{itemize}
\tightlist
\item
  \(\beta_x\): \textbf{Tasa de mortalidad base} para el grupo de edad
  \(x\) en el período base. Representa la mortalidad para cada grupo de
  edad suponiendo que no hubiera cambios en el tiempo.
\end{itemize}

\hypertarget{estimaciuxf3n-1}{%
\subsubsection{Estimación}\label{estimaciuxf3n-1}}

\[
\beta_x = \frac{1}{T} \sum_{t=1}^{T} \log(m_{x,t})
\]

Representa la tasa base de mortalidad para cada grupo de edad en el
periodo base, que por lo general es el periodo inicial. Muestra cómo es
la mortalidad para cada edad dentro de un periodo en específico. Se
calcula a partir de la media de las tasas de mortalidad para cada grupo
de edad en todos los años que se tengan en los datos.

\clearpage

\hypertarget{paruxe1metro-kappa}{%
\subsubsection{Parámetro kappa}\label{paruxe1metro-kappa}}

\begin{itemize}
\tightlist
\item
  \(\kappa_t\): \textbf{Componente de interacción entre edad y tiempo},
  describe cómo las tasas de mortalidad de cada grupo de edad cambian a
  través del tiempo.
\end{itemize}

\hypertarget{estimaciuxf3n-2}{%
\subsubsection{Estimación}\label{estimaciuxf3n-2}}

Este parámetro captura la manera en que las tasas de mortalidad para un
grupo de edad específico cambian con el tiempo, es decir, cómo es que
evoluciona la mortalidad para cada grupo en función de los años.

\[
\kappa_{x,t} = \log(m_{x,t}) - \beta_x - \alpha_t
\]

El conjunto de estas componentes hace el modelo Lee-Carter

\[
\log(m_{x,t}) = \alpha_x + \beta_x \kappa_t + \epsilon_{x,t}
\]

De esta manera hemos visto la estimación de los parámetros que conforman
este modelo.

\hypertarget{plan-de-simulaciones-para-el-segundo-parcial}{%
\subsection{6.9. Plan de simulaciones para el segundo
parcial}\label{plan-de-simulaciones-para-el-segundo-parcial}}

El objetivo de las simulaciones será evaluar la capacidad predictiva del
modelo Lee-Carter y compararlo con un modelo de mortalidad dinámico que
refleje mejor la realidad de la mortalidad y que no tenga las
limitaciones que tenga el modelo Lee-Carter.

Para el segundo parcial, además de realizar las simulaciones
correspondientes, se pretende comparar el modelo Lee-Carter con modelos
dinámicos que reflejen mejor la realidad. Modelos de Cohorte Dinámica o
modelos de aprendizaje automático. También se pretende comparar las
proyecciones realizadas con las proyecciones que realizan las
instituciones como el INEGI o la CONAPO.

\clearpage

\hypertarget{resultados}{%
\section{7. Resultados}\label{resultados}}

\hypertarget{anuxe1lisis-exploratorio-de-datos-1}{%
\subsection{7.1 Análisis exploratorio de
datos}\label{anuxe1lisis-exploratorio-de-datos-1}}

Antes de ajustar el modelo Lee-Carter a los datos demográficos de
México, se realizó un análisis exploratorio de las tasas de mortalidad
(mx) y los años vividos (Lx) por edad y sexo. Este análisis permitió
identificar tendencias importantes y las diferencias que existen entre
hombres y mujeres.

\begin{Shaded}
\begin{Highlighting}[]
\FunctionTok{library}\NormalTok{(ggplot2)}
\CommentTok{\# Códigos para comparar los resultados del analisis exploratorio }
\FunctionTok{library}\NormalTok{(ggplot2)}
\FunctionTok{library}\NormalTok{(dplyr)}

\CommentTok{\# Podemos agregar una columna "sexo" con el fin de identificar hombres y mujeres}

\CommentTok{\# HOMBRES}

\NormalTok{base\_final\_hombres2 }\OtherTok{\textless{}{-}}\NormalTok{ merge16}

\FunctionTok{str}\NormalTok{(base\_final\_hombres2)}
\end{Highlighting}
\end{Shaded}

\begin{verbatim}
## 'data.frame':    646 obs. of  8 variables:
##  $ year: int  1895 1895 1895 1895 1895 1895 1895 1895 1895 1895 ...
##  $ age : int  0 1 5 10 15 20 25 30 35 40 ...
##  $ mx  : chr  "0.30741" "0.23314" "0.0136" "0.00742" ...
##  $ qx  : chr  "0.26465" "0.20796" "0.06567" "0.03643" ...
##  $ lx  : chr  "100000" "73535" "58242" "54417" ...
##  $ Lx  : chr  "82268" "253645" "280694" "267329" ...
##  $ Tx  : chr  "2872238" "2789970" "2536324" "2255631" ...
##  $ ex  : chr  "28.72" "37.94" "43.55" "41.45" ...
\end{verbatim}

\begin{Shaded}
\begin{Highlighting}[]
\CommentTok{\# Verificacion de los datos }
\NormalTok{columnas\_con\_coma2 }\OtherTok{\textless{}{-}} \FunctionTok{sapply}\NormalTok{(base\_final\_hombres2, }\ControlFlowTok{function}\NormalTok{(x) }\FunctionTok{any}\NormalTok{(}\FunctionTok{grepl}\NormalTok{(}\StringTok{","}\NormalTok{,x)))}
\FunctionTok{print}\NormalTok{(columnas\_con\_coma2)}
\end{Highlighting}
\end{Shaded}

\begin{verbatim}
##  year   age    mx    qx    lx    Lx    Tx    ex 
## FALSE FALSE FALSE FALSE  TRUE  TRUE  TRUE FALSE
\end{verbatim}

Realizamos el mismo procedimiento, ahora con las mujeres

\begin{Shaded}
\begin{Highlighting}[]
\CommentTok{\# Eliminamos las comas}
\NormalTok{base\_final\_mujeres2}\SpecialCharTok{$}\NormalTok{lxm }\OtherTok{\textless{}{-}} \FunctionTok{gsub}\NormalTok{(}\StringTok{","}\NormalTok{,}\StringTok{""}\NormalTok{,base\_final\_mujeres2}\SpecialCharTok{$}\NormalTok{lx)}

\CommentTok{\# Detectamos la presencia de mas comas en columnas Tx y ex}
\CommentTok{\# Columna Tx}
\NormalTok{base\_final\_mujeres2}\SpecialCharTok{$}\NormalTok{Txm }\OtherTok{\textless{}{-}} \FunctionTok{gsub}\NormalTok{(}\StringTok{","}\NormalTok{,}\StringTok{""}\NormalTok{,base\_final\_mujeres2}\SpecialCharTok{$}\NormalTok{Tx)}

\CommentTok{\# Columna ex}
\NormalTok{base\_final\_mujeres2}\SpecialCharTok{$}\NormalTok{Lxm }\OtherTok{\textless{}{-}} \FunctionTok{gsub}\NormalTok{(}\StringTok{","}\NormalTok{,}\StringTok{""}\NormalTok{,base\_final\_mujeres2}\SpecialCharTok{$}\NormalTok{Lx)}

\FunctionTok{library}\NormalTok{(corrplot)}
\FunctionTok{str}\NormalTok{(base\_final\_mujeres2)}
\end{Highlighting}
\end{Shaded}

\begin{verbatim}
## 'data.frame':    646 obs. of  8 variables:
##  $ year: int  1895 1895 1895 1895 1895 1895 1895 1895 1895 1895 ...
##  $ age : int  0 1 5 10 15 20 25 30 35 40 ...
##  $ mxm : chr  "0.27222" "0.22613" "0.01243" "0.00619" ...
##  $ qxm : chr  "0.23831" "0.20238" "0.06021" "0.03052" ...
##  $ lxm : chr  "100000" "76169" "60754" "57096" ...
##  $ Lxm : chr  "84510" "263995" "293708" "281295" ...
##  $ Txm : chr  "3093286" "3008776" "2744781" "2451073" ...
##  $ exm : chr  "30.93" "39.5" "45.18" "42.93" ...
\end{verbatim}

\begin{Shaded}
\begin{Highlighting}[]
\NormalTok{base\_final\_mujeres2}\SpecialCharTok{$}\NormalTok{year }\OtherTok{\textless{}{-}} \FunctionTok{as.numeric}\NormalTok{(base\_final\_mujeres2}\SpecialCharTok{$}\NormalTok{year)}
\NormalTok{base\_final\_mujeres2}\SpecialCharTok{$}\NormalTok{age }\OtherTok{\textless{}{-}} \FunctionTok{as.numeric}\NormalTok{(base\_final\_mujeres2}\SpecialCharTok{$}\NormalTok{age)}
\NormalTok{base\_final\_mujeres2}\SpecialCharTok{$}\NormalTok{mxm }\OtherTok{\textless{}{-}} \FunctionTok{as.numeric}\NormalTok{(base\_final\_mujeres2}\SpecialCharTok{$}\NormalTok{mx)}
\NormalTok{base\_final\_mujeres2}\SpecialCharTok{$}\NormalTok{qxm }\OtherTok{\textless{}{-}} \FunctionTok{as.numeric}\NormalTok{(base\_final\_mujeres2}\SpecialCharTok{$}\NormalTok{qx)}
\NormalTok{base\_final\_mujeres2}\SpecialCharTok{$}\NormalTok{lxm }\OtherTok{\textless{}{-}} \FunctionTok{as.numeric}\NormalTok{(base\_final\_mujeres2}\SpecialCharTok{$}\NormalTok{lx)}
\NormalTok{base\_final\_mujeres2}\SpecialCharTok{$}\NormalTok{Lxm }\OtherTok{\textless{}{-}} \FunctionTok{as.numeric}\NormalTok{(base\_final\_mujeres2}\SpecialCharTok{$}\NormalTok{Lx)}
\NormalTok{base\_final\_mujeres2}\SpecialCharTok{$}\NormalTok{Txm }\OtherTok{\textless{}{-}} \FunctionTok{as.numeric}\NormalTok{(base\_final\_mujeres2}\SpecialCharTok{$}\NormalTok{Tx)}
\NormalTok{base\_final\_mujeres2}\SpecialCharTok{$}\NormalTok{exm }\OtherTok{\textless{}{-}} \FunctionTok{as.numeric}\NormalTok{(base\_final\_mujeres2}\SpecialCharTok{$}\NormalTok{ex)}

\FunctionTok{cor}\NormalTok{(base\_final\_mujeres2[, }\FunctionTok{c}\NormalTok{(}\StringTok{"year"}\NormalTok{,}\StringTok{"age"}\NormalTok{,}\StringTok{"mxm"}\NormalTok{,}\StringTok{"qxm"}\NormalTok{,}\StringTok{"lxm"}\NormalTok{,}\StringTok{"Lxm"}\NormalTok{,}\StringTok{"Txm"}\NormalTok{,}\StringTok{"exm"}\NormalTok{)])}
\end{Highlighting}
\end{Shaded}

\begin{verbatim}
##               year           age        mxm        qxm        lxm        Lxm
## year  1.000000e+00 -5.232687e-20 -0.4124693 -0.2529917  0.7015152  0.6924270
## age  -5.232687e-20  1.000000e+00  0.3776712  0.6427863 -0.6104144 -0.3538126
## mxm  -4.124693e-01  3.776712e-01  1.0000000  0.8187782 -0.5828893 -0.6765897
## qxm  -2.529917e-01  6.427863e-01  0.8187782  1.0000000 -0.7245574 -0.6363457
## lxm   7.015152e-01 -6.104144e-01 -0.5828893 -0.7245574  1.0000000  0.8162452
## Lxm   6.924270e-01 -3.538126e-01 -0.6765897 -0.6363457  0.8162452  1.0000000
## Txm   4.969108e-01 -8.154238e-01 -0.4310481 -0.5720546  0.8031022  0.5804541
## exm   3.588688e-01 -8.998722e-01 -0.4845706 -0.6268922  0.7503286  0.5465190
##             Txm        exm
## year  0.4969108  0.3588688
## age  -0.8154238 -0.8998722
## mxm  -0.4310481 -0.4845706
## qxm  -0.5720546 -0.6268922
## lxm   0.8031022  0.7503286
## Lxm   0.5804541  0.5465190
## Txm   1.0000000  0.9773942
## exm   0.9773942  1.0000000
\end{verbatim}

\begin{Shaded}
\begin{Highlighting}[]
\FunctionTok{str}\NormalTok{(base\_final\_mujeres2)}
\end{Highlighting}
\end{Shaded}

\begin{verbatim}
## 'data.frame':    646 obs. of  8 variables:
##  $ year: num  1895 1895 1895 1895 1895 ...
##  $ age : num  0 1 5 10 15 20 25 30 35 40 ...
##  $ mxm : num  0.27222 0.22613 0.01243 0.00619 0.00925 ...
##  $ qxm : num  0.2383 0.2024 0.0602 0.0305 0.0452 ...
##  $ lxm : num  100000 76169 60754 57096 55353 ...
##  $ Lxm : num  84510 263995 293708 281295 270752 ...
##  $ Txm : num  3093286 3008776 2744781 2451073 2169778 ...
##  $ exm : num  30.9 39.5 45.2 42.9 39.2 ...
\end{verbatim}

Ahora, vamos a juntar ambas bases de datos para poder graficar las
curvas de mortalidad por sexo y por edades.

\begin{Shaded}
\begin{Highlighting}[]
\CommentTok{\# Ya podemos crear los objetos que necesitamos }

\NormalTok{base\_final\_hombres2 }\OtherTok{\textless{}{-}}\NormalTok{ base\_final\_hombres2 }\SpecialCharTok{\%\textgreater{}\%} \FunctionTok{mutate}\NormalTok{(}\AttributeTok{sexo =} \StringTok{"Hombres"}\NormalTok{)}
\NormalTok{base\_final\_mujeres2 }\OtherTok{\textless{}{-}}\NormalTok{ base\_final\_mujeres2 }\SpecialCharTok{\%\textgreater{}\%} \FunctionTok{mutate}\NormalTok{(}\AttributeTok{sexo =} \StringTok{"Mujeres"}\NormalTok{)}

\CommentTok{\# Unimos bases que hemos creado }
\NormalTok{datos\_mortalidad }\OtherTok{\textless{}{-}} \FunctionTok{bind\_rows}\NormalTok{(base\_final\_hombres2, base\_final\_mujeres2)}

\CommentTok{\# Elegimos edades}
\NormalTok{edades\_mostrar }\OtherTok{\textless{}{-}} \FunctionTok{c}\NormalTok{(}\DecValTok{0}\NormalTok{,}\DecValTok{10}\NormalTok{,}\DecValTok{20}\NormalTok{,}\DecValTok{30}\NormalTok{,}\DecValTok{40}\NormalTok{,}\DecValTok{50}\NormalTok{,}\DecValTok{60}\NormalTok{,}\DecValTok{70}\NormalTok{,}\DecValTok{80}\NormalTok{)}

\NormalTok{datos\_filtrados }\OtherTok{\textless{}{-}}\NormalTok{ datos\_mortalidad }\SpecialCharTok{\%\textgreater{}\%} \FunctionTok{filter}\NormalTok{(age }\SpecialCharTok{\%in\%}\NormalTok{ edades\_mostrar)}
\end{Highlighting}
\end{Shaded}

Graficamos las curvas de mortalidad para hombres y para mujeres, tomando
en cuenta las edades que definimos anteriormente.

\hypertarget{hombres-1}{%
\subsection{Hombres}\label{hombres-1}}

\begin{Shaded}
\begin{Highlighting}[]
\FunctionTok{ggplot}\NormalTok{(datos\_filtrados, }\FunctionTok{aes}\NormalTok{(}\AttributeTok{x =}\NormalTok{ year, }\AttributeTok{y =} \FunctionTok{log}\NormalTok{(mx), }\AttributeTok{color =}\NormalTok{ sexo)) }\SpecialCharTok{+}
  \FunctionTok{geom\_line}\NormalTok{(}\AttributeTok{size =} \DecValTok{1}\NormalTok{) }\SpecialCharTok{+}
  \FunctionTok{facet\_wrap}\NormalTok{(}\SpecialCharTok{\textasciitilde{}}\NormalTok{ age, }\AttributeTok{scales =} \StringTok{"free\_y"}\NormalTok{) }\SpecialCharTok{+}
  \FunctionTok{labs}\NormalTok{(}
    \AttributeTok{title =} \StringTok{"Comparación exploratoria: tasas log(mx) hombres"}\NormalTok{,}
    \AttributeTok{x =} \StringTok{"Año"}\NormalTok{, }\AttributeTok{y =} \StringTok{"Logaritmo de la tasa de mortalidad"}
\NormalTok{  ) }\SpecialCharTok{+}
  \FunctionTok{theme\_minimal}\NormalTok{()}
\end{Highlighting}
\end{Shaded}

\includegraphics{Proyecto-LC-completo_files/figure-latex/unnamed-chunk-25-1.pdf}

\hypertarget{mujeres-1}{%
\subsection{Mujeres}\label{mujeres-1}}

\begin{Shaded}
\begin{Highlighting}[]
\FunctionTok{ggplot}\NormalTok{(datos\_filtrados, }\FunctionTok{aes}\NormalTok{(}\AttributeTok{x =}\NormalTok{ year, }\AttributeTok{y =} \FunctionTok{log}\NormalTok{(mxm), }\AttributeTok{color =}\NormalTok{ sexo)) }\SpecialCharTok{+}
  \FunctionTok{geom\_line}\NormalTok{(}\AttributeTok{size =} \DecValTok{1}\NormalTok{) }\SpecialCharTok{+}
  \FunctionTok{facet\_wrap}\NormalTok{(}\SpecialCharTok{\textasciitilde{}}\NormalTok{ age, }\AttributeTok{scales =} \StringTok{"free\_y"}\NormalTok{) }\SpecialCharTok{+}
  \FunctionTok{labs}\NormalTok{(}
    \AttributeTok{title =} \StringTok{"Comparación exploratoria: tasas log(mx) mujeres"}\NormalTok{,}
    \AttributeTok{x =} \StringTok{"Año"}\NormalTok{, }\AttributeTok{y =} \StringTok{"Logaritmo de la tasa de mortalidad"}
\NormalTok{  ) }\SpecialCharTok{+}
  \FunctionTok{theme\_minimal}\NormalTok{()}
\end{Highlighting}
\end{Shaded}

\includegraphics{Proyecto-LC-completo_files/figure-latex/unnamed-chunk-26-1.pdf}

\hypertarget{interpretaciuxf3n-de-las-gruxe1ficas}{%
\subsection{Interpretación de las
gráficas}\label{interpretaciuxf3n-de-las-gruxe1ficas}}

\begin{itemize}
\item
  En ambos sexos, se observa una disminución sostenida de las tasas de
  mortalidad a lo largo del tiempo en la mayoría de los grupos de edad,
  esto refleja mejoras significativas en salúd pública, acceso a los
  servicios médicos y una mejora en las condiciones de vida.
\item
  En edades tempranas (0-20 años) las tasas de mortalidad disminuyen de
  una manera drástica en hombres y mujeres.
\item
  En edades intermedias (30-80 años), los hombres presentan tasas de
  mortalidad más elevadas en comparación con las mujeres, lo cual
  refleja un patrón estructural de mayor vulnerabilidad masculina ante
  enfermedades crónicas y riesgos en general.
\item
  Podemos apreciar que, a partir del año 2000, en varias edades
  (especialmente en edades avanzadas), se percibe un ligero aumento en
  las tasas de mortalidad, esto puede estar relacionado con efectos
  recientes como la pandemia de COVID-19 que vivimos unos años atrás.
\end{itemize}

\clearpage

\hypertarget{aplicaciuxf3n-del-modelo-lee-carter}{%
\subsection{7.2. Aplicación del modelo
Lee-Carter}\label{aplicaciuxf3n-del-modelo-lee-carter}}

Procederemos a visualizar y a comentar los resultados obtenidos
posterior a la aplicación del modelo Lee-Carter

\begin{Shaded}
\begin{Highlighting}[]
\CommentTok{\# Cargar librerías necesarias}
\FunctionTok{library}\NormalTok{(demography)}
\FunctionTok{library}\NormalTok{(reshape2)}

\CommentTok{\# Crear el objeto demográfico para hombres usando Lx como población}
\NormalTok{my\_data }\OtherTok{\textless{}{-}} \FunctionTok{demogdata}\NormalTok{(}
  \AttributeTok{data =}\NormalTok{ mx\_matrix,     }\CommentTok{\# Matriz de tasas de mortalidad}
  \AttributeTok{pop =}\NormalTok{ Lx\_matrix,      }\CommentTok{\# Matriz de años vividos}
  \AttributeTok{ages =} \FunctionTok{c}\NormalTok{(}\DecValTok{0}\NormalTok{, }\DecValTok{1}\NormalTok{, }\DecValTok{5}\NormalTok{, }\DecValTok{10}\NormalTok{, }\DecValTok{15}\NormalTok{, }\DecValTok{20}\NormalTok{, }\DecValTok{25}\NormalTok{, }\DecValTok{30}\NormalTok{, }\DecValTok{35}\NormalTok{, }\DecValTok{40}\NormalTok{, }\DecValTok{45}\NormalTok{, }\DecValTok{50}\NormalTok{, }\DecValTok{55}\NormalTok{, }\DecValTok{60}\NormalTok{, }\DecValTok{65}\NormalTok{, }\DecValTok{70}\NormalTok{, }\DecValTok{75}\NormalTok{, }\DecValTok{80}\NormalTok{, }\DecValTok{85}\NormalTok{),}
  \AttributeTok{years =} \FunctionTok{c}\NormalTok{(}\DecValTok{1895}\NormalTok{,}\DecValTok{1897}\NormalTok{,}\DecValTok{1900}\NormalTok{,}\DecValTok{1905}\NormalTok{,}\DecValTok{1910}\NormalTok{,}\DecValTok{1915}\NormalTok{,}\DecValTok{1921}\NormalTok{,}\DecValTok{1925}\NormalTok{,}\DecValTok{1930}\NormalTok{,}\DecValTok{1935}\NormalTok{,}\DecValTok{1940}\NormalTok{,}\DecValTok{1945}\NormalTok{,}
            \DecValTok{1955}\NormalTok{,}\DecValTok{1965}\NormalTok{,}\DecValTok{1975}\NormalTok{,}\DecValTok{1985}\NormalTok{,}\DecValTok{1995}\NormalTok{,}\DecValTok{2005}\NormalTok{,}\DecValTok{2006}\NormalTok{,}\DecValTok{2007}\NormalTok{,}\DecValTok{2008}\NormalTok{,}\DecValTok{2009}\NormalTok{,}\DecValTok{2010}\NormalTok{,}\DecValTok{2011}\NormalTok{,}
            \DecValTok{2012}\NormalTok{,}\DecValTok{2013}\NormalTok{,}\DecValTok{2014}\NormalTok{,}\DecValTok{2015}\NormalTok{,}\DecValTok{2016}\NormalTok{,}\DecValTok{2017}\NormalTok{,}\DecValTok{2018}\NormalTok{,}\DecValTok{2019}\NormalTok{,}\DecValTok{2020}\NormalTok{,}\DecValTok{2021}\NormalTok{),}
  \AttributeTok{type =} \StringTok{"mortality"}\NormalTok{,}
  \AttributeTok{label =} \StringTok{"Tasas de Mortalidad {-} México"}\NormalTok{,}
  \AttributeTok{name =} \StringTok{"Hombres"}
\NormalTok{)}

\CommentTok{\# Ajuste del modelo Lee{-}Carter}
\NormalTok{fit\_lc }\OtherTok{\textless{}{-}} \FunctionTok{lca}\NormalTok{(my\_data, }\AttributeTok{series =} \StringTok{"Hombres"}\NormalTok{, }\AttributeTok{adjust =} \StringTok{"e0"}\NormalTok{)}

\CommentTok{\# Resumen numérico de resultados (varianza explicada, errores, etc.)}
\FunctionTok{summary}\NormalTok{(fit\_lc)}
\end{Highlighting}
\end{Shaded}

\begin{verbatim}
## Lee-Carter analysis
## 
## Call: lca(data = my_data, series = "Hombres", adjust = "e0") 
## 
## Adjustment method: e0
## Region: Tasas de Mortalidad - México
## Years in fit: 1895 - 2021
## Ages in fit: 0 - 85 
## 
## Percentage variation explained: 97.8%
## 
## ERROR MEASURES BASED ON MORTALITY RATES
## 
## Averages across ages:
##       ME      MSE      MPE     MAPE 
## -0.00056  0.00011  0.03198  0.12531 
## 
## Averages across years:
##       IE      ISE      IPE     IAPE 
## -0.03518  0.00696  2.21575  9.81128 
## 
## 
## ERROR MEASURES BASED ON LOG MORTALITY RATES
## 
## Averages across ages:
##      ME     MSE     MPE    MAPE 
## 0.00843 0.03643 0.00141 0.02939 
## 
## Averages across years:
##      IE     ISE     IPE    IAPE 
## 0.66557 2.60205 0.07310 2.29206
\end{verbatim}

\begin{Shaded}
\begin{Highlighting}[]
\CommentTok{\# Gráfico de los componentes ax, bx, kt}
\FunctionTok{plot}\NormalTok{(fit\_lc)}
\end{Highlighting}
\end{Shaded}

\includegraphics{Proyecto-LC-completo_files/figure-latex/resultados-modelo-lc-hombres-1.pdf}

\begin{Shaded}
\begin{Highlighting}[]
\CommentTok{\# Extraer vectores }
\NormalTok{ax }\OtherTok{\textless{}{-}}\NormalTok{ fit\_lc}\SpecialCharTok{$}\NormalTok{ax}
\NormalTok{bx }\OtherTok{\textless{}{-}}\NormalTok{ fit\_lc}\SpecialCharTok{$}\NormalTok{bx}
\NormalTok{kt }\OtherTok{\textless{}{-}}\NormalTok{ fit\_lc}\SpecialCharTok{$}\NormalTok{kt}


\CommentTok{\# Cargar librerias}
\FunctionTok{library}\NormalTok{(demography)}
\FunctionTok{library}\NormalTok{(reshape2)}

\CommentTok{\# Crear el objeto demográfico para mujeres usando Lx como población}
\NormalTok{my\_data\_mujeres }\OtherTok{\textless{}{-}} \FunctionTok{demogdata}\NormalTok{(}
  \AttributeTok{data =}\NormalTok{ mxm\_matrix,     }\CommentTok{\# Matriz de tasas de mortalidad de mujeres}
  \AttributeTok{pop =}\NormalTok{ Lxm\_matrix,      }\CommentTok{\# Matriz de años vividos de mujeres}
  \AttributeTok{ages =} \FunctionTok{c}\NormalTok{(}\DecValTok{0}\NormalTok{, }\DecValTok{1}\NormalTok{, }\DecValTok{5}\NormalTok{, }\DecValTok{10}\NormalTok{, }\DecValTok{15}\NormalTok{, }\DecValTok{20}\NormalTok{, }\DecValTok{25}\NormalTok{, }\DecValTok{30}\NormalTok{, }\DecValTok{35}\NormalTok{, }\DecValTok{40}\NormalTok{, }\DecValTok{45}\NormalTok{, }\DecValTok{50}\NormalTok{, }\DecValTok{55}\NormalTok{, }\DecValTok{60}\NormalTok{, }\DecValTok{65}\NormalTok{, }\DecValTok{70}\NormalTok{, }\DecValTok{75}\NormalTok{, }\DecValTok{80}\NormalTok{, }\DecValTok{85}\NormalTok{),}
  \AttributeTok{years =} \FunctionTok{c}\NormalTok{(}\DecValTok{1895}\NormalTok{,}\DecValTok{1897}\NormalTok{,}\DecValTok{1900}\NormalTok{,}\DecValTok{1905}\NormalTok{,}\DecValTok{1910}\NormalTok{,}\DecValTok{1915}\NormalTok{,}\DecValTok{1921}\NormalTok{,}\DecValTok{1925}\NormalTok{,}\DecValTok{1930}\NormalTok{,}\DecValTok{1935}\NormalTok{,}\DecValTok{1940}\NormalTok{,}\DecValTok{1945}\NormalTok{,}
            \DecValTok{1955}\NormalTok{,}\DecValTok{1965}\NormalTok{,}\DecValTok{1975}\NormalTok{,}\DecValTok{1985}\NormalTok{,}\DecValTok{1995}\NormalTok{,}\DecValTok{2005}\NormalTok{,}\DecValTok{2006}\NormalTok{,}\DecValTok{2007}\NormalTok{,}\DecValTok{2008}\NormalTok{,}\DecValTok{2009}\NormalTok{,}\DecValTok{2010}\NormalTok{,}\DecValTok{2011}\NormalTok{,}
            \DecValTok{2012}\NormalTok{,}\DecValTok{2013}\NormalTok{,}\DecValTok{2014}\NormalTok{,}\DecValTok{2015}\NormalTok{,}\DecValTok{2016}\NormalTok{,}\DecValTok{2017}\NormalTok{,}\DecValTok{2018}\NormalTok{,}\DecValTok{2019}\NormalTok{,}\DecValTok{2020}\NormalTok{,}\DecValTok{2021}\NormalTok{),}
  \AttributeTok{type =} \StringTok{"mortality"}\NormalTok{,}
  \AttributeTok{label =} \StringTok{"Tasas de Mortalidad {-} México"}\NormalTok{,}
  \AttributeTok{name =} \StringTok{"Mujeres"}
\NormalTok{)}

\CommentTok{\# Ajuste del modelo Lee{-}Carter para mujeres}
\NormalTok{fit\_lc\_mujeres }\OtherTok{\textless{}{-}} \FunctionTok{lca}\NormalTok{(my\_data\_mujeres, }\AttributeTok{series =} \StringTok{"Mujeres"}\NormalTok{, }\AttributeTok{adjust =} \StringTok{"e0"}\NormalTok{)}

\CommentTok{\# Resumen numérico de resultados (varianza explicada, errores, etc.)}
\FunctionTok{summary}\NormalTok{(fit\_lc\_mujeres)}
\end{Highlighting}
\end{Shaded}

\begin{verbatim}
## Lee-Carter analysis
## 
## Call: lca(data = my_data_mujeres, series = "Mujeres", adjust = "e0") 
## 
## Adjustment method: e0
## Region: Tasas de Mortalidad - México
## Years in fit: 1895 - 2021
## Ages in fit: 0 - 85 
## 
## Percentage variation explained: 98.5%
## 
## ERROR MEASURES BASED ON MORTALITY RATES
## 
## Averages across ages:
##       ME      MSE      MPE     MAPE 
## -0.00054  0.00010  0.03130  0.11586 
## 
## Averages across years:
##       IE      ISE      IPE     IAPE 
## -0.03506  0.00633  2.37309  9.37368 
## 
## 
## ERROR MEASURES BASED ON LOG MORTALITY RATES
## 
## Averages across ages:
##      ME     MSE     MPE    MAPE 
## 0.01360 0.03026 0.00061 0.02705 
## 
## Averages across years:
##      IE     ISE     IPE    IAPE 
## 1.11345 2.30286 0.01271 2.12253
\end{verbatim}

\begin{Shaded}
\begin{Highlighting}[]
\CommentTok{\# Gráfico de los componentes ax, bx, kt para mujeres}
\FunctionTok{plot}\NormalTok{(fit\_lc\_mujeres)}
\end{Highlighting}
\end{Shaded}

\includegraphics{Proyecto-LC-completo_files/figure-latex/resultados-modelo-lc-hombres-2.pdf}

\begin{Shaded}
\begin{Highlighting}[]
\CommentTok{\# Extraer vectores individuales si los necesitas para análisis comparativo}
\NormalTok{ax\_m }\OtherTok{\textless{}{-}}\NormalTok{ fit\_lc\_mujeres}\SpecialCharTok{$}\NormalTok{ax}
\NormalTok{bx\_m }\OtherTok{\textless{}{-}}\NormalTok{ fit\_lc\_mujeres}\SpecialCharTok{$}\NormalTok{bx}
\NormalTok{kt\_m }\OtherTok{\textless{}{-}}\NormalTok{ fit\_lc\_mujeres}\SpecialCharTok{$}\NormalTok{kt}
\end{Highlighting}
\end{Shaded}

\hypertarget{interpretaciuxf3n-de-los-resultados}{%
\subsection{Interpretación de los
resultados:}\label{interpretaciuxf3n-de-los-resultados}}

Para los resultados, retomaremos aquellos que abordamos en el punto
número 5, en donde se explicó la metodología utilizada en este proyecto:

\hypertarget{hombres-2}{%
\subsection{Hombres}\label{hombres-2}}

\begin{Shaded}
\begin{Highlighting}[]
\FunctionTok{plot}\NormalTok{(fit\_lc)}
\end{Highlighting}
\end{Shaded}

\includegraphics{Proyecto-LC-completo_files/figure-latex/unnamed-chunk-27-1.pdf}

\hypertarget{interpretaciones-de-las-gruxe1ficas-obtenidas}{%
\subsection{Interpretaciones de las gráficas
obtenidas:}\label{interpretaciones-de-las-gruxe1ficas-obtenidas}}

\begin{itemize}
\item
  \(\alpha_x\): \textbf{Nivel promedio de mortalidad}. Este componente
  del modelo Lee-Carter representa la estructura base de la mortalidad
  por edad a lo largo del periodo del análisis. Observamos que la tasa
  de mortalidad aumenta significativamente con la edad, esto se debe al
  envejecimiento natural. En particular, los hombres presentan una
  mortalidad más alta en edades avanzadas en comparación con las
  mujeres. Esto refleja que, históricamente, los hombres han estado más
  expuestos a factores de riesgo o han tenido un menor acceso a
  condiciones de salud en edades más avanzadas.
\item
  \(\beta_x\): \textbf{Sensibilidad al cambio temporal por edad}. El
  parámetro beta en el modelo Lee-Carter refleja el impacto del cambio
  temporal en la mortalidad de cada grupo de edad. La curva para los
  hombres indica que la mortalidad en los jóvenes ha disminuido,
  mientras que para las edades más avanzadas, hay un descenso en las
  tasas de mortalidad, lo que nos indica que son menos pronunciadas en
  los últimos años.
\end{itemize}

\clearpage

\begin{itemize}
\tightlist
\item
  \(\kappa_x\): \textbf{Tendencia temporal de mortalidad}. El índice
  kappa nos muestra cómo ha cambiado el nivel general de mortalidad a lo
  largo del tiempo en nuestro país. Se puede notar una tendencia
  descendente clara y continua desde principios del siglo XX, lo cual es
  un indicador de una mejora sostenida en las condiciones de vida, el
  acceso a servicios de salud, los avances médicos y las políticas
  públicas en México.
\end{itemize}

Esta evolución confirma que el modelo Lee-Carter captura de manera
adecuada la transición epidemiológica en México y el impacto que ha
tenido en la mortalidad de la población.

\hypertarget{mujeres-2}{%
\subsection{Mujeres}\label{mujeres-2}}

\begin{Shaded}
\begin{Highlighting}[]
\FunctionTok{plot}\NormalTok{(fit\_lc\_mujeres)}
\end{Highlighting}
\end{Shaded}

\includegraphics{Proyecto-LC-completo_files/figure-latex/unnamed-chunk-28-1.pdf}

\hypertarget{interpretaciones-de-las-gruxe1ficas-obtenidas-1}{%
\subsection{Interpretaciones de las gráficas
obtenidas:}\label{interpretaciones-de-las-gruxe1ficas-obtenidas-1}}

\begin{itemize}
\item
  \(\alpha_x\): \textbf{Nivel promedio de mortalidad}. La tasa de
  mortalidad aumenta significativamente con la edad, pero de manera
  menos pronunciada en comparación con la de los hombres. Sin embargo,
  el aumento menos pronunciado y los valores son más bajos en todas las
  edades, esto nos indica una mayor esperanza de vida y menor mortalidad
  histórica de las mujeres. Esta esperanza puede estar ligada a factores
  biológicos, sociales y de comportamiento que favorecen a la población
  femenina.
\item
  \(\beta_x\): \textbf{Sensibilidad al cambio temporal por edad}. La
  curva de mortalidad para las mujeres presenta una distribución más
  suave y estable, lo cual nos sugiere que los cambios temporales en la
  mortalidad han sido más uniformes entre los distintos grupos de edad.
  Esto se relaciona con las mejoras en salud pública y condiciones de
  vida que han impactado de manera positiva en la población femenina.
  También podemos notar que este no se concentra fuertemente en ciertos
  grupos como ocurre en el caso de los hombres.
\item
  \(\kappa_x\): \textbf{Tendencia temporal de mortalidad}. La tendencia
  temporal de la mortalidad para el caso de las mujeres muestra una
  disminución progresiva y sostenida de la mortalidad a través del
  tiempo. Esta evolución más moderada puede estar influenciada por
  factores sociales, mayor acceso a los servicios de salud u otras
  condiciones que históricamente han favorecido una mayor longevidad de
  la población femenina.
\end{itemize}

\hypertarget{reflexiones-breves-del-modelo-lee-carter}{%
\subsection{7.3. Reflexiones breves del modelo
Lee-Carter}\label{reflexiones-breves-del-modelo-lee-carter}}

\begin{itemize}
\item
  En el caso del nivel promedio de la mortalidad, observamos que la
  mortalidad aumenta con la edad en ambos sexos, pero este incremento es
  más notorio en el caso de los hombres, lo cual refleja una mayor
  vulnerabilidad masculina durante edades más avanzadas.
\item
  En cuanto a la sensibilidad al cambio temporal por edad, muestra que
  las mujeres han tenido una disminución más uniforme, mientras que en
  el caso de los hombres podemos apreciar fluctuaciones más marcadas en
  ciertas edades.
\item
  La tendencia temporal de mortalidad nos muestra que ambos sexos
  muestran una clara mejora en cuanto a la mortalidad a lo largo del
  tiempo, aunque la caída en el sexo femenino es más estable, esto nos
  sugiere que existen mejores condiciones de salud para el caso de las
  mujeres.
\end{itemize}

\clearpage

\hypertarget{discusiuxf3n}{%
\section{8. Discusión}\label{discusiuxf3n}}

El presente estudio tuvo como objetivo central analizar la evolución de
la mortalidad en México a lo largo de más de un siglo, mediante la
implementación del modelo Lee-Carter. Este enfoque permitió identificar
patrones históricos en las tasas de mortalidad, diferenciados por sexo y
edad, y facilitó la proyección de su comportamiento futuro. La
aplicación del modelo ofreció una herramienta estadística sólida para
descomponer la dinámica de la mortalidad en los componentes que integran
el modelo, permitiendo así una comprensión más profunda de los cambios
estructurales en la salud de la población mexicana.

Los resultados obtenidos tras el ajuste del modelo Lee-Carter
evidenciaron su eficacia para capturar la tendencia decreciente de la
mortalidad en México, especialmente a partir del siglo XX. En
particular, se observaron tasas de mortalidad más elevadas entre los
hombres de edades avanzadas, acompañadas de descensos significativos en
las edades jóvenes en las décadas recientes. En contraste, las mujeres
mostraron consistentemente niveles de mortalidad más bajos, con una
evolución más estable a través del tiempo. Esta diferenciación por sexo
no solo responde a factores biológicos, sino también a determinantes
sociales, como el acceso desigual a servicios de salud, las condiciones
de vida, y el impacto de políticas públicas diferenciadas por el género.

Estas diferencias no solo reflejan el papel que juegan las condiciones
biológicas y sociales, sino también podemos apreciar la desigualdad en
el acceso a los servicios de salud en México, los cambios en los estilos
de vida y los cambios en las políticas públicas. Los patrones observados
pueden ser útiles para proyectar expectativas de vida y de esta manera
diseñar esquemas de pensiones, seguros de vida, planes personales de
retiro o programas de salud que beneficien a las y los mexicanos en un
futuro.

Los patrones identificados aportan información valiosa para la
estimación de la esperanza de vida y la planificación de sistemas de
seguridad social. Estos hallazgos pueden ser utilizados para diseñar
esquemas actuariales más realistas en pensiones, seguros de vida,
programas de salud, entre otros. El modelo permite, por tanto, una base
empírica para el diseño de políticas públicas con enfoque preventivo y
sostenible.

No obstante, es importante reconocer las limitaciones inherentes al
modelo Lee-Carter. Suponer independencia entre edades y la utilización
de un único componente temporal puede resultar insuficiente para
capturar fenómenos complejos como crisis sanitarias (por ejemplo,
pandemias), cambios demográficos abruptos o desigualdades regionales.
Asimismo, el modelo no incorpora explícitamente factores como el nivel
socioeconómico, diferencias geográficas, ni causas específicas de
mortalidad, lo que restringe su aplicación en distintos contextos.

Ya con los resultados del modelo y viendo el panorama general que tiene
México, sería valioso incorporar modelos multivariantes o estocásticos
que permitan capturar las incertidumbres, así como ampliar este análisis
a nivel estatal o incluir las causas específicas de muerte.

Finalmente, una limitación operativa identificada en el desarrollo del
estudio fue la escasa accesibilidad a bases de datos completas y de
calidad. A pesar de los esfuerzos de instituciones como el INEGI o la
CONAPO, la disponibilidad y estandarización de información demográfica
histórica en formatos reutilizables continúa siendo un reto para los
investigadores. Facilitar el acceso a estos datos es crucial para
promover la transparencia y fortalecer la toma de decisiones basada en
evidencia científica.

\clearpage

\hypertarget{conclusiones-1}{%
\section{9. Conclusiones}\label{conclusiones-1}}

A lo largo de este estudio se analizaron las tendencias de mortalidad en
la población mexicana mediante el modelo de mortalidad de Lee-Carter,
integrando información histórica por más de un siglo. El análisis
exploratorio que se realizó al inicio nos permitió identificar patrones
claros en las tasas de mortalidad por edad y por sexo.

Al aplicar el modelo Lee-Carter, se logró descomponer la evolución de la
mortalidad en tres componentes clave: los efectos por edad
(\(\alpha_x\)), la tendencia temporal común (\(\kappa_x\)) y la
sensibilidad por edad a los cambios en el tiempo (\(\beta_x\)). Esto nos
permitió entender con mayor claridad cómo han cambiado los patrones de
mortalidad en México, destacando una reducción general en las tasas y
diferencias estructurales por sexo.

Para trabajos futuros, se recomienda extender el análisis mediante
modelos más flexibles como el modelo Lee-Miller, incorporar covariables
adicionales o realizar un análisis por subgrupos poblacionales. También
podría integrarse el uso de técnicas bayesianas o de aprendizaje
automático (como las redes neuronales) con el propósito de mejorar la
capacidad predictiva de este modelo.

En conjunto, los resultados obtenidos permiten fortalecer la comprensión
de la dinámica demográfica del país, y proporcionan herramientas útiles
para la toma de decisiones en ámbitos actuariales, de salud pública y
políticas de seguridad social en México.

\clearpage

\hypertarget{bibliografuxeda}{%
\section{10. Bibliografía}\label{bibliografuxeda}}

\begin{enumerate}
\def\labelenumi{\arabic{enumi}.}
\tightlist
\item
  Mortality Forecasting in Latin America: An Assessment of the
  Lee-Carter Model
\end{enumerate}

\href{https://www.tandfonline.com/doi/epdf/10.1080/03461238.2021.1928542?needAccess=true}{\emph{Mortality
Forecasting in Latin America: An Assessment of the Lee-Carter Model}}

Janssen, F., \& Debón, A. (2021). An age-specific evaluation of the
Lee-Carter mortality forecasting model. Scandinavian Actuarial Journal,
2021(8), 654-679. \url{https://doi.org/10.1080/03461238.2021.192854}

\begin{enumerate}
\def\labelenumi{\arabic{enumi}.}
\setcounter{enumi}{1}
\tightlist
\item
  Latin American Mortality Database
\end{enumerate}

\href{https://www.ssc.wisc.edu/cdha/latinmortality2/?page_id=28}{\emph{Latin
American Mortality Database}}

University of Wisconsin-Madison. (n.d.). Latin mortality project (2nd
ed.). Center for Demography and Ecology.
\url{https://www.ssc.wisc.edu/cdha/latinmortality2/}

\begin{enumerate}
\def\labelenumi{\arabic{enumi}.}
\setcounter{enumi}{2}
\item
  The Latin American Mortality Database (LAMBdA). Methodological
  Document Version II: August 2021
\item
  Mortality forecasting at age 65 and above: an age-specific evaluation
  of the Lee-Carter model
\end{enumerate}

Lee, R. D., \& Carter, L. (1992). Modeling and forecasting U.S.
mortality. Journal of the American Statistical Association, 87(419),
659-671. \url{https://doi.org/10.1080/03461238.2021.1928542}

\href{https://www.tandfonline.com/doi/full/10.1080/03461238.2021.1928542\#d1e148}{\emph{Mortality
forecasting at age 65 and above: an age-specific evaluation of the
Lee-Carter model}}

\begin{enumerate}
\def\labelenumi{\arabic{enumi}.}
\setcounter{enumi}{4}
\tightlist
\item
  Mortallity Modelling and forecasting: a review of methods
\end{enumerate}

\href{https://www.actuaries.org/CTTEES_TFM/Documents/Booth_Tickle.pdf}{\emph{Mortality
Modelling and forecasting: a review of methods}}

Booth, H., \& Tickle, L. (2008). Mortality modelling and forecasting: A
review of methods. Actuarial Association of Australia. Retrieved from
\url{https://www.actuaries.org/CTTEES_TFM/Documents/Booth_Tickle.pdf}

\end{document}
